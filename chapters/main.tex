\RequirePackage[final]{ifdraft}
\RequirePackage{xstring,xparse}
\NewDocumentCommand\paper{mm}{\IfStrEq{\papersize}{#1}{#2}{}}
\PassOptionsToPackage{usenames,dvipsnames}{xcolor}
\PassOptionsToPackage{quiet}{fontspec}
\documentclass[fontsize=10pt]{scrbook}
%% Packages
% Packages included by <skrapport>:
% polyglossia, microtype, icomma, amsmath,
% unicode-math, xunicode, fontspec
% subfiles
\usepackage{subfiles}
% graphics
%\usepackage[usenames,dvipsnames]{xcolor}
\usepackage{graphicx}
\usepackage{tikz}
\usetikzlibrary{shadows}
% page size
\paper{a4}{%
	\usepackage[a4paper,xetex]{geometry}%
	% a4 is not intended for printing, so we disable crop marks
%	\usepackage[cam,axes,info,b4,center]{crop}%
}
\paper{c5}{%
	\usepackage[c5paper,xetex]{geometry}%
	\usepackage[cam,axes,info,b5,center]{crop}%
}
% code
\usepackage{minted}
% improvement packages
\usepackage{multirow}
\usepackage[para,perpage,ragged,bottom,stable]{footmisc}
\usepackage{enumitem}
\usepackage{fancyhdr}
\usepackage{booktabs}
\usepackage[multiple]{fnpct}
% references
\usepackage{csquotes}
\usepackage[style=authoryear,sorting=nyt,sortcites=false,%
			language=swedish,babel=other,urldate=iso8601,%
			mergedate=basic,dashed=false,%
			backref=true,abbreviate=false,dateabbrev=false,%
			bibstyle=chapters/custom-authoryear]{biblatex}
% other stuff
\usepackage{siunitx}
\usepackage{hologo}
\let\sups\relax
\usepackage[safe,noenc]{tipa}
\usepackage[titles]{tocloft}
\usepackage[font=small,format=plain,labelfont=bf,textfont=it]{caption}
\usepackage{subfig}
\usepackage{sidecap}
\usepackage{cprotect}
\usepackage{bookmark}
\usepackage{xspace}
\usepackage{xcoffins}
\usepackage{mathtools}
\usepackage{varioref}
\usepackage{hyperref}
\usepackage[nameinlink,noabbrev]{cleveref}
\AtBeginDocument{
	\hypersetup{
		hyperfootnotes=true,
		linkcolor=red!75!black,
		citecolor=.,
		anchorcolor=.,
		filecolor=.,
		urlcolor=magenta
	}
}
\paper{a4}{\hypersetup{colorlinks=true}}
\paper{c5}{\hypersetup{colorlinks=false}}
% Source code line numbers in margin
%\usepackage{srcline}
% Paket som demonstreras i del 6
\usepackage{braket}
\usepackage{xfrac}
% CRITICAL BUGFIX FOR unicode-math (see https://github.com/wspr/unicode-math/issues/227)
\makeatletter
\def\@cdots{\mathinner{\unicodecdots}}
\makeatother

%% Draft fixes to minted, tikz
\ExplSyntaxOn
\ifdraft{%
	\cs_if_exist_use:cT{define@key}{{FV}{mathescape}{}}
	% Replace minted with regular environments
	\RenewDocumentCommand\mint{omv}{%
		\texttt{#3}
	}
	\RenewDocumentEnvironment{minted}{om}{%
		\VerbatimEnvironment%
		\fvset{#1}%
		\begin{Verbatim}%
	}{%
		\end{Verbatim}%
	}
	\RenewDocumentCommand\inputminted{omm}{%
		\fvset{#1}%
		\VerbatimInput{#3}%
	}
	% LaTeX source code
	\newminted{latex}{frame=single}%
	\newminted{text}{frame=single}%
	\newminted{r}{frame=single}%
	\newminted{matlab}{frame=single}%
	\newminted{gnuplot}{frame=single}%
	\newmint{latex}{frame=single}%
	% Line numbers in margin
	%\lnoson% % Disable when not spellchecking/proofreading
}{%
	% Common options
	\cs_set_nopar:Nn\l__tmpa_cs:{
		frame=single,
		bgcolor=mintedbg,
		rulecolor=\color{mintedbg}
	}
	% Define environments
	\exp_args:Nnx\newminted{latex}{\l__tmpa_cs:}
	\exp_args:Nnx\newminted{text}{\l__tmpa_cs:}
	\exp_args:Nnx\newminted{r}{\l__tmpa_cs:}
	\exp_args:Nnx\newminted{matlab}{\l__tmpa_cs:}
	\exp_args:Nnx\newminted{gnuplot}{\l__tmpa_cs:}
	\exp_args:Nnx\newmint{latex}{\l__tmpa_cs:}
	% Undefine the temporary macro
	\cs_undefine:N\l__tmpa_cs:
}
\ExplSyntaxOff

%% FONT SELECTION
\ExplSyntaxOn

%% Set the font directory
\cs_if_exist:NF\latexbokFontdir
	{\cs_gset:Npn\latexbokFontdir{fonts}}

%% Font features
\defaultfontfeatures{
	Numbers={Proportional,OldStyle},
	Ligatures={Historic,Common,Required,Contextual},
	Contextuals=Swash,
	SmallCapsFeatures={
		Letters=SmallCaps,
		Scale=0.96,
		Weight=0.8
	},
}
%% Main font: Alegreya
\setmainfont[Scale=1.05,
		Path=\latexbokFontdir/alegreya/,
		ItalicFont=Alegreya-Italic.ttf,
		BoldFont=Alegreya-Bold.ttf,
		BoldItalicFont=Alegreya-BoldItalic.ttf
	]{Alegreya-Regular.ttf}
%% Sans serif font: Open Sans
\setsansfont[Scale=0.9,
		Path=\latexbokFontdir/open-sans/,
		ItalicFont=OpenSans-Italic.ttf,
		BoldFont=OpenSans-Bold.ttf,
		BoldItalicFont=OpenSans-BoldItalic.ttf
	]{OpenSans-Regular.ttf}
%% Monospace font: Consola mono
\setmonofont[Scale=0.9,
		Path=\latexbokFontdir/consola-mono/,
		BoldFont=ConsolaMono-Bold.ttf,
	]{ConsolaMono-Regular.ttf}
%% Title font: Avería Serif
\newfontfamily\titlefont[Scale=1.05,
		Path=\latexbokFontdir/averia-serif/,
		ItalicFont=AveriaSerif-Italic.ttf,
		BoldFont=AveriaSerif-Bold.ttf,
		BoldItalicFont=AveriaSerif-BoldItalic.ttf
	]{AveriaSerif-Regular.ttf}

%% UTF and TIPA fonts
\cs_set_eq:NN\utffont\rmfamily
\cs_set_eq:NN\tipafont\sffamily

%% Patch the title style to include the new font
\cs_generate_variant:Nn\cs_set_protected:Nn{No}
\cs_set_protected:No\__skrapport_title_style:
	{\__skrapport_title_style:\exp_not:N\titlefont}

\ExplSyntaxOff

%% Hacks
\makeatletter
% Fix \hologo{XeTeX}
%\def\HoLogo@Xe#1{% 
%   X% 
%   \kern-.1em\relax 
%   \ltx@IfUndefined{HOLOGO@ReflectBox\hologodriver} 
%     {\ltx@IfUndefined{HOLOGO@ReflectBox}\ltx@firstoftwo\ltx@secondoftwo} 
%     \ltx@secondoftwo 
%   {e}{% 
%     \lower.5ex\hbox{% 
%       \HOLOGO@ReflectBox{E}% 
%     }% 
%   }% 
%}
% Force cramped style on inline math
\def\(#1\){\relax\ifmmode\@badmath\else$\fi\cramped{#1}\relax\ifmmode\ifinner $\else\@badmath\fi\else\@badmath\fi}
%\def\(#1\){\relax\ifmmode\@badmath\else$\fi\smash{#1}\relax\ifmmode\ifinner $\else\@badmath\fi\else\@badmath\fi}
% Varioref seems broken with XeLaTeX
\def\reftextfaceafter{på \reftextvario{motstående}{nästa} sida}%
\def\reftextfacebefore{på \reftextvario{motstående}{föregående} sida}%
\def\reftextafter{på \reftextvario{följande}{nästa} sida}%
\def\reftextbefore{på föregående sida}%
\def\reftextcurrent{på denna sida}%
\def\reftextfaraway#1{{på sidan~\pageref{#1}}}%
\def\reftextpagerange#1#2{{på sidorna~\pageref{#1}–\pageref{#2}}}%
\def\reftextlabelrange#1#2{{\ref{#1} till~\ref{#2}}}%
\DeclareRobustCommand\Vref{\@ifstar{\let\vref@space\relax\Vr@f}{\let\vref@space~\Vr@f}}
\DeclareRobustCommand\vref{\@ifstar{\let\vref@space\relax\vr@f}{\let\vref@space~\vr@f}}
% Twoside
\@twosidetrue
% Listings
\newcommand{\kodname}{Exempel}
\newcounter{kod}\renewcommand\thekod{\@arabic\c@kod}
\def\fps@kod{tb}
\def\ftype@kod{3}
\def\ext@kod{lok}
\def\fnum@kod{\kodname~\thekod}
\newenvironment{kod}{\@float{kod}}{\end@float}
\newenvironment{kod*}{\@dblfloat{kod}}{\end@dblfloat}
\newsubfloat{kod}
% Displaying one line of code
\newcommand\kodrad[2]{\hspace{1ex}\inputminted[firstline=#1,lastline=#1]{latex}{#2.tex}}
% Redefining sections to break page
\let\section@old=\section
\renewcommand\section{\@ifstar\my@section@star\my@section}
\newcommand\my@section[2][\@empty]{\cleardoublepage\vspace*{2\bigskipamount}\ifx\@empty#1\section@old{#2}\else\section@old[#1]{#2}\fi}
\newcommand\my@section@star[2][\@empty]{\cleardoublepage\vspace*{2\bigskipamount}\ifx\@empty#1\section@old*{#2}\else\section@old*[#1]{#2}\fi}
% Front/main/backmatter
\newcommand\frontmatter{%
  \cleardoublepage
  \pagenumbering{roman}}
\newcommand\mainmatter{%
  \cleardoublepage
  \pagenumbering{arabic}}
\newcommand\backmatter{%
  \cleardoublepage
}
% Maketitle without newpage
\newcommand\maketitlennp{\par%
  \begingroup
    \renewcommand\thefootnote{\@fnsymbol\c@footnote}%
    \def\@makefnmark{\rlap{\@textsuperscript{\normalfont\@thefnmark}}}%
    \long\def\@makefntext##1{\parindent 1em\noindent%
      \hb@xt@1.8em{\hss\@textsuperscript{\normalfont\@thefnmark}}##1}%
    \global\@topnum\z@
    \@maketitlennp
    \thispagestyle{empty}\@thanks
  \endgroup
  \setcounter{footnote}{0}
}
\def\@maketitlennp{%
  \null
  \begin{flushleft}%
\vspace{-\headsep}
    {\small%
      \if\@regarding\relax\else\@regarding{, }\fi%
      \@date\par%
    }%
    \vspace{1.5cm}%
    {\Huge\@titstyle\@title\par}%
    \vspace{.125cm}%
    {\Large\@titstyle\@author}%
    \vspace{.75cm}%
  \end{flushleft}%
  \par%
}
% Changes to titlepage environment (don't reset pagenumbering before)
\renewenvironment{titlepage}{\cleardoublepage}{\thispagestyle{skrapport@titlepage}\cleardoublepage\setcounter{page}\@ne}
% TexorPDF version of LaTeX and AmS logo
\let\@oldLaTeX\LaTeX
\def\LaTeX{\texorpdfstring{\@oldLaTeX}{LaTeX}}
\let\@oldAmS\AmS
\def\AmS{\texorpdfstring{\@oldAmS}{AMS}}
% File last modified
\def\parsedate #1:2#2#3#4#5#6#7#8#9\empty{\ifx{#2}{9}19\else20\fi#3#4/#5#6/#7#8}
\newcommand\moddate[1][\jobname.tex]{\expandafter\parsedate\pdffilemoddate{#1}\empty}
% Minted background fix
\renewenvironment{minted@colorbg}[1]{%
	\par%
	\def\minted@bgcol{#1}%
	\noindent\begin{lrbox}{\minted@bgbox}%
	\begin{minipage}{\linewidth-2\fboxsep}%
}{%
	\end{minipage}%
	\end{lrbox}%
	\colorbox{\minted@bgcol}{\usebox{\minted@bgbox}}%
	\par%
}
\makeatother
%% Convenient definitions
\newcommand\PDF{\textsc{PDF}\xspace}					% PDF format
\newcommand\PNG{\textsc{PNG}\xspace}					% PNG format
\newcommand\DVI{\textsc{DVI}\xspace}					% DVI format
\newcommand\EPS{\textsc{EPS}\xspace}					% EPS format
\newcommand\JPEG{\textsc{JPEG}\xspace}                  % JPEG format
\newcommand\UTF{\textsc{UTF-8}\xspace}					% UTF-8 standard
\newcommand\cli[1]{\texttt{#1}}							% Command-line input
\newcommand\opt[1]{\texttt{\emph{#1}}}					% CLI option
\newcommand\pack[1]{\textsf{#1}}						% LaTeX package
\newcommand\pdfLaTeX{\hologo{pdfLaTeX}\xspace}			% pdfLaTeX logotype
\newcommand\BibTeX{\textsc{Bib}\TeX\xspace}				% BibTeX logotype
\newcommand\TikZ{Ti\textit{k}Z\xspace}                  % TikZ logotype
\newcommand\PGFTikZ{PGF/\TikZ\xspace}				    % PGF/TikZ logotype
\newcommand\XeTeX{\hologo{XeTeX}\xspace}				% XeTeX logotype
\newcommand\LyX{L\!\raisebox{-.5ex}{Y}\!X\xspace}		% LyX logotype
\newcommand\MATLAB{\textsc{Matlab©}\xspace}             % MATLAB© logotype
\newcommand\gnuplot{gnuplot\xspace}                     % Gnuplot logotype
\newcommand\Rlogo{R\xspace}                             % R logotype
\newcommand\Mathematica{Mathematica\xspace}             % Mathematica logotype
\newcommand\eng[1]{(eng.~\textenglish{\emph{#1})}}		% english translation
\newcommand\cmd[1]{\textenglish{\texttt{\textbackslash{}#1}}}% LaTeX command
\newcommand\env[1]{\textenglish{\texttt{#1}}}			% LaTeX environment
% minted style
\usemintedstyle{solarized}
\definecolor{mintedbg}{rgb}{0.9665,0.9550,0.9175}
% Unit definitions
\DeclareSIUnit\point{pt}
\DeclareSIUnit\em{em}
\DeclareSIUnit\mu{mu}
\sisetup{locale=DE}
% Math redefinitions
\DeclareMathOperator{\erfc}{erfc}
\renewcommand{\frac}[2]{\genfrac{}{}{}{}{\displaystyle #1}{\displaystyle #2}}
\makeatletter
\renewcommand\d[1]{\ensuremath{\;\mathrm{d}#1\@ifnextchar\d{\!}{}}}
\makeatother
% Useful colors
\definecolor{required}{rgb}{0.384,0.576,0.000}
\definecolor{optional}{rgb}{0.992,0.843,0.000}
\definecolor{unavailable}{rgb}{0.863,0.196,0.184}

%% Redefinitions of style
\ExplSyntaxOn
% Twoside
\paper{a4}{\KOMAoptions{twoside=false,open=any}}
\paper{c5}{\KOMAoptions{twoside=true,open=right}}
% Section numbering depth
\setcounter{secnumdepth}{2}
% Table of contents depth
\setcounter{tocdepth}{1}
% Table of contents bf sections
\RenewDocumentCommand\cftsecfont{}{\bfseries}
\ExplSyntaxOff


%% Load the bibliography data
\addbibresource{referenser.bib}

%% METADATA
\title{Att \hologo{TeX}a: en praktisk guide}
\author{Simon Sigurdhsson}
\subject{Nybörjarguide i \hologo{LaTeX}, andra upplagan.}
\date{}

%% Actual document
\begin{document}
	%% Title page
	\subfile{chapters/meta/titlepage.tex}
	\cleardoublepage 
	%% Table of contents
	\frontmatter
	\begingroup
		\ExplSyntaxOn
		\cs_set_eq:NN\__old_pack\pack
		\cs_set_eq:NN\__old_cmd\cmd
		\cs_set_eq:NN\__old_env\env
		\DeclareDocumentCommand\pack{som}{\textsf{#3}}
		\DeclareDocumentCommand\cmd{som}{\texttt{\textbackslash{}#3}}
		\DeclareDocumentCommand\env{som}{\texttt{#3}}
		\ExplSyntaxOff
		\tableofcontents
		\ExplSyntaxOn
		\cs_set_eq:NN\pack\__old_pack
		\cs_set_eq:NN\cmd\__old_cmd
		\cs_set_eq:NN\env\__old_env
		\ExplSyntaxOff
	\endgroup
	\cleardoublepage
	\mainmatter
	%% INLEDNING
	\subfile{chapters/0/0.tex}
	%% GRUNDLÄGGANDE BEGREPP
	\part{Grunderna}
	\subfile{chapters/1/1.tex}
	%% TYPSÄTTNING MED LATEX
	\subfile{chapters/2/2.tex}
	%% MATEMATIK MED LATEX OCH AMS
	\part{Matematiken}
	\subfile{chapters/3/3.tex}
	%% GRAFIK MED LATEX
	\part{Figurerna}
	\subfile{chapters/4/4.tex}
	%% REFERENSER MED BIBTEX
	\part{Referenserna}
	\subfile{chapters/5/5.tex}
	%% VIDARE LÄSNING
	\part{Fortsättningen}
	\subfile{chapters/6/6.tex}
	%% BIBLIOGRAFI
	\cleardoublepage
	\paper{a4}{\newcommand\latexbokurlfont{\small}}
	\paper{c5}{\newcommand\latexbokurlfont{\footnotesize\ttfamily}}
	\defbibnote{ctan-info}{%
		\protect\index{CTAN}%
		Nedan följer en litteraturförteckning. I denna listas alla de
		böcker, artiklar och manualer som refererats till i boken, med
		hänvisningar till de sidor verket nämndes på. En del av de verk
		som listas, främst olika manualer, har en CTAN-länk istället för
		ett ISBN-nummer eller en URL. Dessa verk kan man hitta genom att
		lägga till \url{http://mirrors.ctan.org/} innan länken, eller
		genom att titta i sin \LaTeX-distribution (och då är det enklast
		att använda \cli{texdoc}, som beskrivs på \cpageref{cli:texdoc}).%
	}
	\printbibliography[heading=bibintoc,prenote=ctan-info]
	%% INDEX
	% korsrefererande index
	% kapitel 0?
	\index{mellanrum|see{tomrum}}
	\index{pdfLaTeX@\pdfLaTeX{}|see{kompilator, \pdfLaTeX{}}}
	\index{LuaLaTeX@\hologo{LuaTeX}|see{kompilator, \hologo{LuaTeX}}}
	\index{XeLaTeX@\hologo{XeTeX}|see{kompilator, \hologo{XeTeX}}}
	% kapitel 1?
	\index{nyradstecken|see{tomrum}}
	\index{och-tecken|see{specialtecken}}
	\index{\&!i text|see{specialtecken}}
	\index{\&!i tabell|see{tabell}}
	\index{procenttecken|see{specialtecken}}
	\index{\%!i text|see{specialtecken}}
	\index{\%!i \LaTeX-kod|see{kommentar}}
	\index{dollartecken|see{specialtecken}}
	\index{\$!i text|see{specialtecken}}
	\index{\$!matematikläge|see{matematikläge}}
	\index{klass|see{dokumentklass}}
	\index{pappersstorlek|see{inställningar, standardklass}}
	\index{installera!paket|see{paket, installera}}
	\index{kompilera!till \PDF|see{\PDF*}}
	\index{kompilera!automatiskt|see{\cli{latexmk}}}
	% Kapitel 2
	\index{kapitel|see{rubrik}}
	\index{radbrytning!automatisk|see{avstavning}}
	\index{kursiv|see{textstil}}
	\index{fetstil|see{textstil}}
	\index{kapitäler|see{textstil}}
	\index{serif-typsnitt|see{typsnitt}}
	\index{sans-serif-typsnitt|see{typsnitt}}
	\index{referens!kors-|see{korsreferens}}
	\index{sakregister|see{register}}
	\index{förteckning|see{register}}
	\index{index!register|see{register}}
	\index{förteckning!innehålls-|see{innehållsförteckning}}
	\index{framsida|see{titelsida}}
	\index{mått!i LaTeX@i \LaTeX{}|see{längd}}
	\index{flytande objekt!figur|see{figur}}
	\index{flytande objekt!tabell|see{tabell}}
	% Kapitel 3
	\index{mått!i text|see{SI-enhet}}
	\index{enhet|see{SI-enhet}}
	\index{index!subskript|see{subskript}}
	\index{exponent|see{superskript}}
	\index{matematikläge!tomrum|see{tomrum}}
	\index{matematikläge|seealso{matematik}}
	\index{ekvation|see{matematikläge}}
	% Kapitel 4
	\index{figur!infoga|see{grafik}}
	\index{pgfTikZ@\PGFTikZ|see{grafik, generera}}
	% Kapitel 5
	\index{referens!-lista|see{bibliografi}}
	\index{Harvard-stil|see{bibliografistil}}
	\index{Chicago-stil|see{bibliografistil}}
	% Kapitel 6
	\index{resurser!CTAN|see{CTAN}}
	\index{resurser!TeX.SE@\TeX.SE|see{\TeX{} Stack Exchange}}
	% Skriv förteckningar
	\KOMAoptions{parskip=never}
	\cleardoublepage
	\indexprologue{\label{sec:idx}
		Detta register innehåller en förteckning över olika teman, termer
		och liknande som diskuteras i boken. Den innehåller inte en
		förteckning över de \LaTeX-paket som diskuteras (en sådan återfinns
		på~\cpageref{sec:idx-pkg}) eller de kommandon och miljöer som nämns
		(en förteckning över dessa finns på~\cpageref{sec:idx-cmd}).
	}
	\printindex
	\cleardoublepage
	\indexprologue{\label{sec:idx-pkg}
		Denna förteckning listar alla de \LaTeX-paket som nämns i boken.
	}
	\printindex[packages]
	\cleardoublepage
	\indexprologue{\label{sec:idx-cmd}
		Denna förteckning listar alla de kommandon och mijöer som 
		diskuteras i boken.
	}
	\printindex[macros]
	\KOMAoptions{parskip=half*}
	%% EN ENKEL MALL
	\cleardoublepage
	\appendix
	\part{Appendix}
	\subfile{chapters/meta/appendix.tex}
	\backmatter
\end{document}
