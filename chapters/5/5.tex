\ifdefined\latexbokFontdir\else\def\latexbokFontdir{../../fonts}\fi
\ifdefined\latexbokFiguredir\else\def\latexbokFiguredir{../../examples}\fi
\documentclass[lang=sv,ptsize=10pt,font=none,nomath,titles=bf,../../a4.tex]{subfiles}
\begin{document}
\section{Referenser med \BibTeX}\label{sec:5}
En stor fördel med \LaTeX{} är att man kan automatisera hanteringen av
till exempel referenser. \BibTeX{} är ett verktyg som gör detta ännu
enklare och som gör att man kan samla alla sina referenser i en extern
databas, vilken kan hanteras av ett externt program\footnote{Exempel på
sådana är \emph{JabRef}, \emph{BibDesk} och \emph{refbase}.\hfill} eller 
(som sig bör när det gäller \LaTeX) med en enkel textredigerare.

Systemet består av stilfiler (med filändelse \texttt{.bst}), databaser
(\texttt{.bib}) och programmet \texttt{bibtex}. Denna introduktion kommer
endast att gå igenom databasformatet och hur denna används från \LaTeX{}; 
en ytterligare introduktion ges av \citeasnoun{Fenn06} och en komplett
manual är även tillgänglig \cite{Patashnik88a}. Att skapa egna stilfiler
involverar en del programmering i ett stackorienterat postfix-språk som
dokumenteras av \pack{btxhak} \cite{Patashnik88b}, men detta är inte att
rekommendera. Bättre är att använda några av de \BibTeX-stilar som finns
på CTAN (till exempel \pack{chscite}, om man vill följa 
Chalmers-bibliotekets rekommendationer).

\subsection{\BibTeX-databasen}
Den databas \BibTeX{} använder är i all sin enkelhet en vanlig textfil.
Denna textfil innehåller ett antal block, ett per referens i databasen,
som innehåller information om referensen (\emph{fält}). En typisk referens
innehåller en nyckel (vilken man sedan använder när man refererar till
referensen i \LaTeX-dokumentet), en titel, en författare och ett årtal.
Dessutom innehåller varje block information om vilken typ av referens 
(bok, artikel och så vidare) det handlar om. 

\begin{kod}[tbp]
	\centering
	\vfil\cprotect[mm]\ifdraft{\inputminted[frame=single]{latex}{\latexbokFiguredir/5/bibdata.bib}}{\inputminted[frame=single,bgcolor=mintedbg,rulecolor=\color{mintedbg}]{latex}{\latexbokFiguredir/5/bibdata.bib}}\vfil
	\caption{En enkel exempelreferens ur en \BibTeX-databas.}
	\label{ex:bibtex}
\end{kod}

Exempel~\vref{ex:bibtex} visar ett block ur en \BibTeX-databas; blocket
inleds med en blocktyp (\texttt{@ARTICLE}) och den nyckel som används då
man refererar till källan från \LaTeX{} (\texttt{Hassanpour08}) varefter
ett antal självförklarande fält med information följer. Olika blocktyper
kräver olika fält, både obligatoriska (till exempel titel och författare)
och frivilliga (i det här fallet är bland annat \texttt{month} frivillig)
— tabell~\vref{tab:blocktyper} listar de vanligaste blocktyperna och de
obligatoriska/frivilliga fält som hör till.

\begin{table}[bp]
	\begin{minipage}{0.98\textwidth}
		\newlength{\oldfboxsep}\setlength{\oldfboxsep}{\fboxsep}
		\setlength{\fboxsep}{0pt}
		\newcommand{\required}{% (fold)
			\setlength{\fboxsep}{\oldfboxsep}%
			\colorbox{required}{\color{required}o}%\checkmark}%
		} % (end)
		\newcommand{\optional}{% (fold)
			\setlength{\fboxsep}{\oldfboxsep}%
			\colorbox{optional}{\color{optional}v}%\Large$\circ$}%
		} % (end)
		\newcommand{\unavailable}{% (fold)
			\setlength{\fboxsep}{\oldfboxsep}%
			\colorbox{unavailable}{\color{unavailable}x}%$\times$}%
		} % (end)
		\centering
		\caption{\setlength{\oldfboxsep}{0.5\oldfboxsep}%
			Vanliga referenstyper i \BibTeX{} och deras tillhörande fält.
			Både typnamn och fältnamn är relativt självförklarande.
			\emph{Nyckel:} \fbox{\required} obligatoriskt,
			\fbox{\optional} valfritt och
			\fbox{\unavailable} otillgängligt fält.
		}
		\label{tab:blocktyper}
		\begin{tabular}{lc@{}c@{}c@{}c@{}c@{}c@{}c@{}c@{}%
						 c@{}c@{}c@{}c@{}c@{}c@{}c@{}c}
		\toprule 
		Blocktyp & % (fold)
			\rotatebox{90}{\texttt{address}} &
			\rotatebox{90}{\texttt{author}} &
			\rotatebox{90}{\texttt{booktitle}} &
			\rotatebox{90}{\texttt{chapter}} &
			\rotatebox{90}{\texttt{edition}} &
			\rotatebox{90}{\texttt{editor}} &
			\rotatebox{90}{\texttt{journal}} &
			\rotatebox{90}{\texttt{month}} &
			\rotatebox{90}{\texttt{number}} &
			\rotatebox{90}{\texttt{pages}} &
			\rotatebox{90}{\texttt{publisher}} &
			\rotatebox{90}{\texttt{school}} &
			\rotatebox{90}{\texttt{series}} &
			\rotatebox{90}{\texttt{title}} &
			\rotatebox{90}{\texttt{volume}} &
			\rotatebox{90}{\texttt{year}} \\
		% (end)
		\midrule 
		\texttt{@article} & \unavailable & \required & \unavailable & \unavailable & \unavailable & \unavailable & \required & \optional & \optional & \optional & \unavailable & \unavailable & \unavailable & \required & \optional & \required \\
		\texttt{@book}\footnote{\label{fn:auth-editor}Antingen \texttt{author} \emph{eller} \texttt{editor} ska specificeras, ej båda två.} & \optional & \required & \unavailable & \unavailable & \optional & \required & \unavailable & \optional & \unavailable & \unavailable & \required & \unavailable & \optional & \required & \optional & \required \\
		\texttt{@booklet} & \optional & \optional & \unavailable & \unavailable & \unavailable & \unavailable & \unavailable & \optional & \unavailable & \unavailable & \unavailable & \unavailable & \unavailable & \required & \unavailable & \optional \\
		\texttt{@inbook}$^\text{\ref{fn:auth-editor}}$\footnote{\label{fn:chap-pages}Det räcker med att specificera ett av fälten \texttt{chapter} och \texttt{pages}.} & \optional & \required & \unavailable & \required & \optional & \required & \unavailable & \optional & \unavailable & \required & \required & \unavailable & \optional & \required & \optional & \required \\
		\texttt{@incollection} & \optional & \required & \required & \unavailable & \unavailable & \optional & \unavailable & \optional & \unavailable & \optional & \optional & \unavailable & \unavailable & \required & \unavailable & \required \\
		\texttt{@inproceedings} & \optional & \required & \required & \unavailable & \unavailable & \optional & \unavailable & \optional & \unavailable & \optional & \optional & \unavailable & \unavailable & \required & \unavailable & \required \\
		\texttt{@manual} & \unavailable & \optional & \unavailable & \unavailable & \optional & \unavailable & \unavailable & \optional & \unavailable & \unavailable & \unavailable & \unavailable & \unavailable & \required & \unavailable & \optional \\
		\texttt{@mastersthesis} & \optional & \required & \unavailable & \unavailable & \unavailable & \unavailable & \unavailable & \optional & \unavailable & \unavailable & \unavailable & \required & \unavailable & \required & \unavailable & \required \\
		\texttt{@misc} & \unavailable & \optional & \unavailable & \unavailable & \unavailable & \unavailable & \unavailable & \optional & \unavailable & \unavailable & \unavailable & \unavailable & \unavailable & \optional & \unavailable & \optional \\
		\texttt{@phdthesis} & \optional & \required & \unavailable & \unavailable & \unavailable & \unavailable & \unavailable & \optional & \unavailable & \unavailable & \unavailable & \required & \unavailable & \required & \unavailable & \required \\
		\texttt{@proceedings} & \optional & \unavailable & \unavailable & \unavailable & \unavailable & \optional & \unavailable & \optional & \unavailable & \unavailable & \optional & \unavailable & \unavailable & \required & \unavailable & \required \\
		\texttt{@unpublished} & \unavailable & \required & \unavailable & \unavailable & \unavailable & \unavailable & \unavailable & \optional & \unavailable & \unavailable & \unavailable & \unavailable & \unavailable & \required & \unavailable & \optional \\
		\bottomrule 
		\end{tabular}
	\end{minipage}
\end{table}

Oftast är det dock lättare att hantera sin referensdatabas med någon form
av dedikerat verktyg istället för att redigera \texttt{.bib}-filerna
för hand med en textredigerare. Programmet JabRef%
\footnote{\url{http://jabref.sourceforge.net/}}
fungerar mycket bra till detta och finns tillgängligt för alla
plattformar.

Många webbaserade sökverktyg för akademiska publikationer, till exempel
Scopus\footnote{\url{http://www.scopus.com/home.url}}, gör det möjligt att
enkelt exportera information om de artiklar man refererar till i \BibTeX-%
format, som sedan kan läggas till direkt i den egna referensdatabasen.
Detta är mycket praktiskt i större projekt så som kandidatuppsatser.

\subsection{\BibTeX{} med \LaTeX{}}
För att referera till de referenser man har i sin referensdatabas
i sitt \LaTeX-dokument krävs ett par olika saker. Först och främst så
måste man specificera en bibliografistil och inkludera sin
databas i dokumentet där man vill ha sin bibliografi:
\begin{latexcode}
\bibliographystyle{plain}
\bibliography{databas}
\end{latexcode}
Notera här att bibliografistilen (mer om dessa\vpageref{sec:5:stilar})
måste specificeras
innan databasen inkluderas, och att databasen inkluderas med sitt namn
\emph{utan filändelsen}; koden ovan inkluderar alltså referenser från
filen \texttt{databas.bib}. Man kan specificera flera olika databaser
genom att separera dem med kommatecken.

När man sedan kompilerar sitt dokument måste man även köra \BibTeX{} på
sitt dokument; detta görs genom att köra kommandot \texttt{bibtex 
"dokument"} (om \TeX-dokumentet heter \texttt{dokument.tex}) — notera
avsaknaden av filändelse även här. Därefter måste man köra \LaTeX{} ännu
en gång för att referenserna ska dyka upp. En normal kompileringsrunda
med \BibTeX{} är alltså \LaTeX$\to$\BibTeX$\to$\LaTeX$\to$\LaTeX. Med
extra verktyg så som \texttt{latexmk} görs detta automatiskt.

\BibTeX{} inkluderar dock endast de referenser som används i dokumentet
(det vill säga sådana som refereras till direkt i texten), så om din
referenslista inte dyker upp ska du inte vara orolig. Det finns ett antal
olika sätt att referera till källor med \BibTeX{}; två stycken inbyggda
(\cmd{cite} och \cmd{nocite}) samt de som definieras av olika paket (till
exempel \pack{harvard}, vilket diskuteras~\vpageref{sec:5:harvard}).

\newpage
Kommandot \cmd{cite} refererar till en källa i databasen och skriver ut
en hänvisning till bibliografin. För de ursprungliga stilarna i \BibTeX{}
är denna hänvisning en siffra eller nyckel inom hakparanteser (alltså
något i stil med ”[1]”), men detta kan modifieras av diverse paket och
bibliografistilar. Kommandot tar ett obligatoriskt argument, nyckeln till
källan i databasen, och ett frivilligt som lägger till text efter
hänvisningen. Det frivilliga argumentet kan alltså användas för att 
hänvisa till specifika sidor eller delar av en källa:
\begin{latexcode}
\cite{Hassanpour08}       % Skriver ut ”[1]”
\cite[s.~5]{Hassanpour08} % Skriver ut ”[1, s. 5]”
\end{latexcode}

Notera här att det obligatoriska argumentet ska motsvara den nyckel som
anges i databasen; här refererar vi till den källa som visas i
exempel~\vref{ex:bibtex}.
Man kan även referera till flera källor samtidigt genom att separera
dessa med kommatecken:
\latex|\cite{Hassanpour08,Khan10} % Skriver ut ”[1,2]”|

Kommandot \cmd{nocite} gör samma sak som \cmd{cite} men skriver inte ut
någon hänvisning. Detta kommando kan därför användas om man vill
inkludera en källa i bibliografin utan att faktiskt referera till den i
text. Här finns endast det obligatoriska argumentet. Vill man inkludera
alla källor i databasen i sitt dokument kan man istället för en nyckel
ange ”\texttt{*}”:
\latex|\nocite{*}|

\subsubsection{Bibliografistilar}\label{sec:5:stilar}
Det finns en uppsjö olika sätt att referera till källor när man skriver
vetenskapliga rapporter, och dessa kan variera kraftigt i stil. Några
använder siffror (IEEE, Vancouver) eller nycklar (\AmS) för att hänvisa 
till källor i text medan andra (Harvard, Chicago, APA) använder sig av
författarnamnen.
\BibTeX, i sitt grundutförande, kan endast skapa bibliografier och
referenser med nycklar eller siffror. Vissa paket så som \pack{harvard}
och \pack{chscite} implementerar dock Harvard- och Chicago-stilen.

\newpage
Även själva bibliografin, alltså listan över källor, kan utformas på
många sätt. Figur~\vref{fig:bibstyle} visar några av de stilar
som finns tillgängliga med \BibTeX{} och på CTAN. Vilken stil man vill
använda är upp till författaren (eller kanske oftare förläggaren), men
generellt sett så brukar man i matematiska kretsar använda 
\texttt{alpha.bst}, i (elektro)ingenjörskretsar \texttt{ieeetr.bst} (som
ej visas i figuren) och i humanistiska vetenskaper Harvard-stilen (eller
någon närliggande stil). Chalmers avviker från detta och rekommenderar att
man använder en Chicago-liknande stil~\cite{ChsLib10}.

\begin{figure}[tp]
	\centering
	\subfloat[\texttt{abbrv.bst}]{\label{fig:bibstyle:abbrv}% (fold)
		\fbox{\includegraphics[width=0.45\textwidth,%
							   trim=0 -14 0 0]{\latexbokFiguredir/5/abbrv.pdf}}
	} \hfil % (end)
	\subfloat[\texttt{alpha.bst}]{% (fold)
		\fbox{\includegraphics[width=0.45\textwidth]{\latexbokFiguredir/5/alpha.pdf}}
	} \\ % (end)
	\subfloat[\texttt{apalike.bst}]{\label{fig:bibstyle:apalike}% (fold)
		\fbox{\includegraphics[width=0.45\textwidth]{\latexbokFiguredir/5/apalike.pdf}}
	} \hfil % (end)
	\subfloat[\texttt{chscite.bst} från \pack{chscite}]{% (fold)
		\label{fig:bibstyle:chscite}%
		\fbox{\includegraphics[width=0.45\textwidth,%
							   trim=0 -22 0 0]{\latexbokFiguredir/5/chscite.pdf}}
	} \\ % (end)
	\subfloat[\texttt{dcu.bst} från \pack{harvard}]{% (fold)
		\fbox{\includegraphics[width=0.45\textwidth,%
							   trim=0 -11.5 0 0]{\latexbokFiguredir/5/harvard-dcu.pdf}}
	} \hfil % (end)
	\subfloat[\texttt{jurabib.bst} från \pack{jurabib}]{% (fold)
		\label{fig:bibstyle:jurabib}%
		\fbox{\includegraphics[width=0.45\textwidth,%
							   trim=0 0 0 5,clip=true]{\latexbokFiguredir
							   /5/jurabib.pdf}}
	} \\ % (end)
	\caption[Några av de bibliografistilar som kan åstadkommas med
	hjälp av \BibTeX.]{Några av de bibliografistilar som kan åstadkommas
	med hjälp av \BibTeX.
	Figurerna~\subref{fig:bibstyle:abbrv}–\subref{fig:bibstyle:apalike} 
	visar standardstilar, medan  
	figurerna~\subref{fig:bibstyle:chscite}–\subref{fig:bibstyle:jurabib} 
	visar stilar som definieras av olika paket.}
	\label{fig:bibstyle}
\end{figure}

\subsubsubsection*{Harvard- och Chicago-stilen: \pack{harvard} och \pack{chscite}}\label{sec:5:harvard}
Harvard- och Chicago-stilarna, varav den senare beskrivs utförligt i
\emph{The Chicago Manual of Style} \cite{Chicago10},
är sätt att referera till källor baserat
på deras författare, och används ofta i de humanistiska vetenskaperna.
Fördelen med dessa stilar är att referenserna smälter in bättre med den
omgivande texten och ger ett bättre sammanhang.

Tyvärr så är \BibTeX{} från början inte gjort för att stödja de här
referensstilarna, varför det har dykt upp ett antal olika paket som
gör det möjligt. Ett av dessa paket är \pack{harvard}, som beskrivs
utförligt av \citeasnoun{Williams08}. Paketet definierar förutom det
vanliga kommandot \cmd{cite} tre relativt självförklarande kommandon:
\cmd{citeasnoun}, som refererar till en källa som ett substantiv,
\cmd{possessivecite}, som refererar till en källa i genitivform,
och \cmd{citeaffixed}, som refererar som \cmd{cite} men med ett tillägg.
Hur dessa bör användas i text illustreras av följande exempel:
\begin{latexcode}
...vilket visades av \citeasnoun{Hassanpour08}.
\possessivecite{Hassanpour08} undersökning visade att...
...flera rapporter \citeaffixed{Hassanpour08}{t.ex.}...
\end{latexcode}
	
\label{sec:chscite}
Chalmers rekommenderar att man använder en stil baserad på Chicago-%
stilen \cite{ChsLib10}, och dessa rekommendationer implementeras av 
paketet \pack{chscite} som bygger på \pack{harvard} och därför fungerar
på samma sätt. Även detta paket finns på CTAN.
\end{document}
