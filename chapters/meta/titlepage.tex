\ifdefined\latexbokFontdir\else\def\latexbokFontdir{../../fonts}\fi
\ifdefined\latexbokFiguredir\else\def\latexbokFiguredir{../../examples}\fi
\documentclass[10pt,../../a4.tex]{subfiles}
\begin{document}
\pagestyle{empty}
\pagenumbering{Alph}

%\begin{titlepage} % INTERNAL TITLE PAGE (fold)
\publishers{
	%Senast uppdaterad \moddate{}.\\
	\ccbysa\\[1ex]
	Licenserad under Creative Commons By-Sa 4.0.\\[1ex]
	\large\url{http://creativecommons.org/licenses/by-sa/4.0/}
}
\uppertitleback{
	\begingroup
	\backtrackerfalse
	Den här boken är
	en inkomplett guide till att skriva och typsätta \LaTeX-dokument riktad
	till studenter på Chalmers Tekniska Högskola, specifikt programmen
	Teknisk Matematik och Teknisk Fysik.
	Inspiration har tagits från bland annat \textcites{Schultz05}{Voss10},
	men främst från \textcite{Oetiker11}.
	\endgroup
}

% DEDICATION, THANKS AND PUBLISHER
\lowertitleback{
	Boken är typsatt med \hologo{XeLaTeX}, och designen är baserad på 
	dokumentklassen \textsf{scrbook}, som finns tillgänglig på
	CTAN som en del av KOMA-Script.
	Typsnitten som används är \emph{Linux Libertine}, \emph{Linux Biolinum} och \emph{Bitstream Vera Sans Mono}.
	Dessa typsnitt finns tillgängliga på
	\url{http://www.linuxlibertine.org} och
	\url{https://www-old.gnome.org/fonts/},
	respektive.
	\\[\bigskipamount]
	\textcopyright 2014 \textsc{Simon Sigurdhsson}.\\
	Detta verk är licensierat under \emph{Creative Commons Erkännande-DelaLika 4.0}. För att ta del av en kopia av licensen, besök \url{http://creativecommons.org/licenses/by-sa/4.0/deed.sv} eller skicka ett brev till \emph{Creative Commons, 171 2nd Street, Suite 300, San Francisco, California, 94105, USA}.
	Källkoden till boken finns tillgänglig på
	\url{http://github.com/urdh/latexbok/}.
	\\[\bigskipamount]
	%Tryck: XXX\\
	Göteborg, 2014.
}
\dedication{
%		{\csname__skrapport_title_style:\endcsname\huge Tillägnad}\\[1ex]
%		någon?
%		\bigskip\bigskip\bigskip
	{\usekomafont{disposition}\Large Tack till}\\[1.5ex]
	\emph{Phaddergrupp 255},\\
	som korrekturläst och\\
	hjälpt till att förbättra boken\\[1.5ex]
	{\usekomafont{disposition}\Large och}\\[1.5ex]
	\emph{Christian von Schultz},\\
	\emph{Tobias Oetiker} och \emph{Herbert Voß}\\
	för inspirerande texter på samma tema.
}

% Force titleback even in onepage mode
\makeatletter
\paper{a4}{
	\@twosidetrue
	\maketitle
	\@twosidefalse
}
\paper{c5}{
	\maketitle
}
\makeatother

% Preface
\chapter*{\prefacename}
\thispagestyle{empty}
\markboth{\MakeMarkcase{\prefacename}}{\MakeMarkcase{\prefacename}}
Denna bok, vars första upplaga publicerades 2011, är i grunden ett försök
både översätta och förbättra \citetitle{Oetiker11}
\parencite{Oetiker11}, och därmed få en introduktion till \LaTeX{} som är
både modern och skriven på svenska. I det tidigaste skedet var tanken att
boken skulle kunna användas som en inofficiell eller informell kursbok
eller referens i datointroduktionen som ges till förstaårsstudenter i
Teknisk Fysik och Teknisk Matematik vid Chalmers Tekniska Högskola, som
innehåller ett antal föreläsningar om \LaTeX{}. Glädjande nog verkar boken
fylla denna roll, och den har dessutom använts som referenslitteratur i
motsvarande kurs på Maskinteknik vid Chalmers.

I likhet med många andra IT-system utvecklas \LaTeX{} ständigt, och därför
är en introduktion som denna inte aktuell hur länge som helst. Först
upplagan av denna bok är förvisso fortfarande någorlunda korrekt i de
flesta avseenden — speciellt de tre första kapitlen som behandlar de
absoluta grunderna — men \LaTeX{} (och min kunskap om detsamma, vilket
illustrerar hur mycket man kan lära genom att bara använda \LaTeX{} när
tillfälle ges) har utvecklats mycket under de tre år som gått sedan boken
först publicerades på internet. Därför började jag under 2013 fila på
förbättringar, både stora och små, av en del avsnitt i boken. De största
förändringarna har skett i kapitel 4, som i förra upplagan var mycket kort
men som numera behandlar \PGFTikZ{} och \pack*{graphicx} mer utförligt
och även diskuterar hur man allmänt skapar bra grafik, och i kapitel 5
som nu behandlar den moderna ersättaren av \BibTeX, \pack*{biblatex}.

Utöver dessa stora förändringar har ett antal små uppdateringar gjorts
i andra kapitel, både för att förtydliga sådant som var otydligt i förra
upplagan, för att uppdatera sådant som blivit utdaterat (detta gäller inte
minst listan över paket i kapitel 6) och för att göra boken mer tillämpbar
för en bredare publik — även om första upplagan till stor del kunde
användas även utanför den tänkta målgruppen bör denna upplaga vara än
mer allmän.

\LaTeX{} utvecklas som sagt med rasande fart, och delar av boken (framför
allt kapitel 6) kommer säkerligen vara utdaterade inom ett par år — förhoppningsvis blir det då en ny upplaga av boken. Tills dess välkomnas
givetvis rapporter om alla sorters errata, förbättringsmöjligheter eller
liknande, som med fördel kan rapporteras som ärenden på Github\footnote{\url{https://github.com/urdh/latexbok}}.

\medskip
\noindent
Simon Sigurdhsson\\
Göteborg, 3 januari 2014
\end{document}
