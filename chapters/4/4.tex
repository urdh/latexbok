\ifdefined\latexbokFontdir\else\def\latexbokFontdir{../../fonts}\fi
\ifdefined\latexbokFiguredir\else\def\latexbokFiguredir{../../examples}\fi
\documentclass[lang=sv,ptsize=10pt,font=none,nomath,titles=bf,../../a4.tex]{subfiles}
\begin{document}
\section{Grafik med \LaTeX}\label{sec:4}
Grafik, i form av figurer och diagram, kan vara ett mycket bra sätt
att illustrera eller beskriva samband, algoritmer och data. Nedan följer
ett antal olika tips, inte bara om hur man inkluderar (eller till och 
med skapar) grafik i \LaTeX, utan även en inledning som diskuterar vad
\emph{bra} grafik är, och hur man på bästa sätt använder grafik.

Den här delen av boken handla således inte uteslutande om \LaTeX\ och
dess ekosystem, utan även om datavisualisering. Den som främst är
intresserad av de \LaTeX-relaterade bitarna kan därför hoppa över
det inledande stycket (även om det är både kort och informativt) och
gå direkt på \pack{graphicx} och \PGFTikZ\ på
\cpageref{sec:includegraphics}.

\subsection{Vad är bra grafik?}
\index{figur}
\index{grafik}
Att skapa bra grafik, grafik som på ett effektivt och tydligt sätt
kommunicerar den information din data innehåller, är inte så självklart
som man kan tro. Det finns oerhöt många variabler, från vilken typ av
grafik du ska välja \parencite{Henry12} till hur du ska sortera din
data \parencites{Camoes10}[\pno~178]{Tufte01} för att förmedla innehållet
på absolut bästa sätt. Ibland kan man till och med ifågasätta om grafik
verkligen behövs, eller om en tabell presenterar datan bättre.
\Citeauthor{Tufte01}, som skrivit många böcker inom ämnet%
\footnote{Tuftes trilogi \emph{\citetitle{Tufte01}} \parencite{Tufte01},
	\emph{\citetitle{Tufte90}} \parencite{Tufte90} och
	\emph{\citetitle{Tufte97}} \parencite{Tufte97} är väl värd att läsa
	för den som är intresserad av grafik och hur man presenterar data.},
sammanfattar bra grafik på ett mycket exakt och kortfattat sätt
\textcite[\pno~51]{Tufte01}:
\begin{quotation}
	Fulländad grafik består av komplicerade idéer kommunicerade med
	klarhet, precision och effektivitet.
\end{quotation}

Precisionen bör vara det enklaste att uppnå. 
Bland annat handlar det om att inte (medvetet eller omedvetet, till
exempel genom inkorrekta relativa skalor eller 3D-effekter)
ljuga om sin data \parencite[\ppno~57\psqq]{Tufte01},
vilket man undviker genom att förse sina figurer med rättvisande,
relevanta axlar och att inte ta data ur sitt sammanhang
\parencite[\ppno~75\psq]{Tufte01}, vilket bör vara enkelt att undvika
genom att inte klippa bort för mycket data.
Även korrekta och relevanta figurtexter hjälper precisionen.

Klarheten är något mer komplicerad att uppnå. Här handlar det bland annat
om att undvika överflödig information \eng{chart junk}, men även om att
använda rätt typ av diagram, bra färgscheman
\parencite[\ppno~75–77]{Tufte97} och rimliga längdskalor
\parencite[\ppno~20–23]{Tufte97}. Även proportioner kan spela roll;
det är till exempel lättare att se föränringar i ett \enquote{långt},
horisontellt linjediagram än i ett \enquote{högt}, vertikalt dito
\parencite[\ppno~186\psq]{Tufte01}.

Effektiviteten är svårast att uppnå. Med hög datatäthet
\parencite[\pno~162]{Tufte01}, det vill säga många datapunkter per
areaenhet (till en rimligt gräns, så att klarheten inte försummas), är
ett sätt att uppnå effektivitet. Att dessutom se till att så mycket av figuren som möjlig förmedlar faktisk data, genom att till exempel göra
figuraxlar mer spartanska eller utelämna rutsystem, kan också öka
grafikens effektivitet \parencite[\ppno~123\psqq]{Tufte01}. 
Effektiviteten är förmodligen den del av \enquote{fulländad} grafik
som är svårast att uppnå i matematisk programvara.

\subsubsection{Allmänna tips}
Även om datavisualisering är ett stort och inte helt trivialt område, 
och därmed kanske överkurs för de som inte skriver böcker eller långa
rapporter, så bör man hålla någon form av miniminivå för att åtminstone
få fram sitt budskap. Man skulle kunna sammanfatta en sådan lägstanivå
i ett par enkla punkter:
\begin{itemize}
	\index{figur}
	\item \emph{Se till att axlar i figurerna är korrekta och
		informativa.} Detta innebär att axlarna bör vara tydligt
		numrerade, att det framgår om en annolunda skala (till exempel en
		logaritmisk sådan) används, och att axlarna har beskrivande
		etiketter (med enheter).
	\index{färgschema}
	\item \emph{Se till att välja lämpliga färgscheman.} Ta till exempel
		färgblindhet eller kontrast i svartvita utskrifter i beaktande.
		För enkla linjediagram och dylikt är det ganska enkelt, men i
		andra fall kan det vara svårt att välja ett färgschema%
		\footnote{I sådana fall är Colorbrewer, tillgängligt på 
		\url{http://colorbrewer2.org}, ett mycket bra verktyg.} som på
		ett bra sätt förmedlar informationen
		\parencite{Borland07,Moreland09a}.
	\item \emph{Undvik otydliga eller missvisande diagram.}
		Tredimensionella linjediagram kan vara oerhört svåra att läsa,
		och cirkeldiagram lider av liknande problem, till exempel. Om
		du illustrerar skillnader med hjälp av figurer av olika storlek,
		kom då ihåg att \emph{arean} är det som läses av, inte höjden.
	\item \emph{Undvik onödiga extradetaljer.} Mönster, fyllda områden,
		skuggor, 3D-effekter och liknande tillför sannolikt ingen
		information, och syftar mest till att förvirra läsaren.
\end{itemize}

\subsubsection{Programspecifika tips: %
			   Bra \Rlogo-figurer med \texttt{ggplot2}}
\label{sec:ggplot2}
Riktlinjerna och tipsen som introducerats ovan kan vara mer eller mindre
komplicerade att implementera när man ritar figurer i programvara. Vissa
program skapar redan bra och informativ grafik utan extra arbete, men
annan programvara kan kräva mer arbete. Ett program av den senare typen
är \Rlogo.

Standardgrafiken i \Rlogo är överlag ganska bra när det gäller statistisk
analys, men bristande på andra områden. På många sätt är det dessutom 
svårt att göra lite mer involverade saker som att ändra färger, sätta
ordentliga axlar, enheter och så vidare. \Rlogo-paketet \texttt{ggplot2}
skapades för att åtgärda detta, och skapar snyggare grafer på enklare
sätt.

Paketet är intuitivt och baserat på logiska kopplingar mellan data och
grafiska element, vilket gör att man kan bygga grafik genom att tänka
på datans innebörd och vad man vill kommunicera, istället för själva 
grafiken. En bra sammanfattning av paketet, som gör sig bäst på
originalspråket, är \foreignquote{british}{makes doing the right thing
easy, while keeping harder things possible}.

Att installera och använda \texttt{ggplot2} är enkelt. Efter att ha 
installerat paketet genom att köra kommandot
\verb|install.packages("ggplot2")|
i \Rlogo använder man det genom att inkludera paketet och använda
funktionen \texttt{ggplot} i kombination med någon av paketets många
plot-funktioner. Man kan även kombinera olika grafer:
\begin{rcode}
require(ggplot2)
# Exempeldata
df <- data.frame(
+   trt = factor(c(1, 1, 2, 2)),
+   resp = c(1, 5, 3, 4), 
+   group = factor(c(1, 2, 1, 2)),
+   se = c(0.1, 0.3, 0.3, 0.2)
+) 
df2 <- df[c(1,3),] 
# Toppen/botten på errorbars
limits <- aes(ymax = resp + se, ymin=resp - se)
dodge <- position_dodge(width=0.9)
# Plotta!
p <- ggplot(df, aes(fill=group, y=resp, x=trt))
p + geom_bar(position=dodge) +
  + geom_errorbar(limits, position=dodge, width=0.25)
\end{rcode}

En mer utförlig referens över alla kommandon i \texttt{ggplot2} finns
på internet\footnote{\url{http://had.co.nz/ggplot2/}}, och en grundlig
genomgång ges av \textcite{Hadley09}.

\subsection{Inkludera grafik}\label{sec:includegraphics}
\index{grafik!inkludera}
Att kunna inkludera grafik av olika slag i sitt dokument kan verka så
vitalt och grundläggande att det rimligtvis inte bör finnas många olika
möjligheter. Så är dock inte fallet. Under den långa tid som \LaTeX{} har
existerat (och utvecklats) har en rad olika metoder för att inkludera
grafik, alla med specifika för- och nackdelar, utvecklats för systemet.

I grunden kan man dock dela upp dessa metoder i två undergrupper: de
paket som inkluderar grafik som skapats av externa program (till
exempel bilder i \PDF-, \PNG- eller \JPEG-format) och de som använder
funktionalitet i den underliggande \TeX-kompilatorn för att skapa grafik.

I den första kategorin finns i princip bara ett alternativ
— \pack{graphicx}. Den senare kategorin har ett antal kandidater,
men för \pdfLaTeX\ och \XeTeX\ finns det en mycket stark kandidat:
\PGFTikZ. Båda dessa, samt deras för- och nackdelar, diskuteras nedan.

\subsubsection{Paketet \pack{graphicx}}
Den metod som oftast används, eftersom den kräver kortare
kompileringstider och mindre direkt ansträngning, är att inkludera
redan existerande bilder i \PDF- eller \PNG-format med hjälp av
paketet \pack{graphicx}. Paketet har ett mycket enkelt gränssnitt, och
har egentligen bara ett kommando av intresse: \cmd{includegraphics}.

Med hjälp av detta kommando kan man inkludera grafik av format som är
kompatibla med den \TeX-kompilator som används. I fallet \pdfLaTeX\ är
dessa format \PNG, \PDF\ och \JPEG. Kommandot tar ett valfritt
argument, en lista av nycklar, som används för att på olika sätt
skala om bilden som inkluderas, och ett obligatoriskt argument som
berättar var bilden finns:
\latex|\includegraphics[width=\textwidth]{filnamn}|
Fler sådana nycklar listas av \textcite[8\psq]{Carlisle05}.

Fördelen med denna metod är att den i princip \emph{alltid} fungerar,
eftersom nästan alla verktyg kan exportera \PNG- eller \PDF-filer (och
i de fall detta inte går kan man oftast konvertera till \PNG eller i
västa fall använda skärmdumpar). Metoden är såklart också bra för att
inkludera fotografier i \JPEG-format, till exempel bilder på
experimentuppställningar.

Nackdelen är att dessa format inte är vektorbaserade. Detta innebär att
om du behöver skala om bilden i efterhand, så kan grafiken bli suddig
eller otydlig. För till exempel linjediagram och liknande kan dessutom
texten på axlarna bli för liten eller otydlig, vilket såklart gör att
informationen inte kommuniceras lika effektivt. 

Sammanfattat kan man säga att \pack{graphicx} är utmärkt för att
inkludera \JPEG-bilder (dvs. fotografier), medans diagram, grafer och
liknande oftast blir bättre med \PGFTikZ eller \pack{pgfplots}.

\subsubsection{Rita med \PGFTikZ}
\index{grafik!generera}
En metod som ofta ger bättre resultat, men som är lite mer avancerad att
använda (både eftersom den ger längre kompileringstider och eftersom den
kräver mer arbete av användaren) är att använda \PGFTikZ\ för att rita
den grafik som behövs.

\PGFTikZ\ är ett (mycket stort) paket som används för att rita
vektorgrafik direkt i \LaTeX. Paketet har en omfattande manual
\parencite{Tantau10} som innehåller många väldokumenterade exempel.
Dessutom ges en bra, kortfattad introduktion av \textcite{Mertz07}.
Använder man ett program som kan exportera \PGFTikZ-kod finns det oftast
ingen anledning att inte göra det, och i vissa fall (så som för enkla
figurer över uppställningar eller mekaniska figurer) kan det vara 
lättare att skriva \PGFTikZ-kod än att rita figuren i ett externt program.
CTAN-katalogen innehåller ett stort antal tilläggspaket till \PGFTikZ%
\footnote{\url{http://www.ctan.org/tex-archive/graphics/pgf/contrib}},
bland annat \pack{circuitikz} \parencite{Redaelli12} som används
för att rita kopplingsscheman, \pack{flowchart} \parencite{Robson13}
som gör det lättare att rita flödesscheman, \pack{pgfgantt}
\parencite{Skala13} för Ganttscheman och \pack{tikz-timing}
\parencite{Scharrer11} som ritar timing-diagram.

Utöver de exempel som finns i manualen går det även att hitta mycket
material på internet\footnote{Bland annat finns galleriet
\TeX{}ample.net\footref{fn:4:texample} som innehåller 
många exempel på \PGFTikZ-kod, och taggen \emph{\mbox{tikz-pgf}} på
\TeX.SE\footref{fn:4:tex-se-tikz-pgf}
där man kan få svar på många frågor}. Ska man bara rita enkla grafer
finns dessutom det lite enklare paketet \pack{pgfplots}, som använder
\PGFTikZ\ internt (och som diskuteras nedan).%
\silentfootnote{\url{http://www.texample.net}\label{fn:4:texample}}%
\silentfootnote{\url{http://tex.stackexchange.com/questions/tagged/tikz-pgf}\label{fn:4:tex-se-tikz-pgf}}

Fördelen med \PGFTikZ ligger främst i det faktum att grafiken är
vektorbaserad, och därför ser lika bra ut oavsett hur mycket den skalas
om, samt att text i figuren behandlas av \LaTeX\ och därför kommer visas
i rätt typsnitt och vettiga storlekar vilket ökar läsbarheten i figuren.

Den främsta nackdelen är att \PGFTikZ-figurer ofta tar en stund att
kompilera, och att det inte alltid är möjligt att exportera figurer
i det formatet. \PGFTikZ-figurer är alltså att föredra för diagram,
grafer och liknande, men går inte att använda för till exempel
fotografier.

\subsubsubsection*{Plotta med \pack{pgfplots}}
Paketet \pack{pgfplots} kan sägas vara ett tillägg till \PGFTikZ som gör
det lättare att rita vanliga diagram (stapel- eller linjediagram, till
exempel) med hjälp av \PGFTikZ. Bland annat kan \pack{pgfplots} läsa in
data från externa filer och automatiskt omvandla den till figurer.

Paketet har en omfattande manual \parencite{Feuersanger13a}, som
detaljerat beskriver hur man använder paketet för att rita olika
sorters diagram.
Bland annat använder \texttt{matlab2tikz} — som diskuteras på
\cpageref{sec:matlab2tikz} — funktionalitet från \pack{pgfplots}.
Det finns även ett påbyggnadspaket, \pack{pgfplotstable}
\parencite{Feuersanger13b}, som kan läsa
CSV-liknande filer och använda \pack{pgfplots} för att rita diagram
(och tabeller med \pack{booktabs}) utifrån dessa data.

\subsection{Exportera grafik eller data}
\index{figur!exportera|(}
När man väl skapat sin grafik i det program man arbetar med och är nöjd
med denna måste man exportera den till ett format som kan användas med
\LaTeX. Använder man \pdfLaTeX (vilket man bör) så stöds formaten \JPEG,
\PNG och \PDF. Av dessa lämpar sig \JPEG bäst till fotografier, medan
\PNG och \PDF lämpar sig väl till figurer och grafer.
Generellt sett är \PDF att föredra då formatet är vektorbaserat och
därför kan skalas både upp och ner utan att man förlorar någon kvalitet.

Om möjligheten finns är det bästa dock att exportera sina figurer som
\PGFTikZ-kod, vilket innebär att \LaTeX{} kommer att rendera dem och att
man därför har något större frihet när det gäller finjustering av figuren
i efterhand, och även att axlar och dylikt typsätts på samma sätt som
dokumentets brödtext vilket ger ett enhetligt intryck. Ett alternativ
till att exportera hela figurer som \PGFTikZ-kod är att exportera datan
i ett format som kan läsas av \pack{pgfplots}, och sedan använda det
paketet för att rita figurerna.

Det finns många (matematiska) programvaror, och det är orimligt att
inkludera instruktioner till alla i detta kapitel. Jag har därför valt
att beskriva hur man enklast exporterar figurer som \PGFTikZ-kod eller
i \PDF-format (i nödfall \PNG) från två vanliga programvaror:
\Rlogo och \MATLAB.

\subsubsection{Exportera i \PGFTikZ-format}
\PGFTikZ brukar inte vara ett format som normalt stöds av kommersiella
(eller fria) programvaror, och därför är funktionen ofta implementerad
som någon form av tillägg. Lyckligtvis brukar dessa vara öppna och fria
att använda för vem som helst.

\subsubsubsection*{Från \Rlogo med \texttt{tikzDevice}}
\Rlogo-paketet \texttt{tikzDevice} \parencite{Sharpsteen12} gör det möjligt
att i \Rlogo exportera figurer som \PGFTikZ-kod. Det kan som många andra
\Rlogo-paket hittas på CRAN och installeras enklast genom att köra
\Rlogo-kommandot \verb|install.packages("tikzDevice")|. Man använder
sedan kommandot \texttt{tikz} för att berätta för \Rlogo att grafiken
ska skrivas som \PGFTikZ-kod, och filen som skapas kan sedan inkluderas
i \LaTeX-dokument med kommandot \cmd{input}.
\Cref{ex:tikzdevice} visar resultatet av detta.

\begin{kod}[tbp]
	\centering
	\begin{minipage}{\textwidth}
		\centering
		\input{\latexbokFiguredir/4/tikzdevice.tex}
	\end{minipage}
	\\[1ex]
	\begin{minipage}{\textwidth}
		\begin{rcode*}{mathescape=true}
require(tikzDevice)
# width och height specificeras i tum
tikz('figur.tex', width=4, height=2.5)
# exempel med ggplot2, som diskuteras på $\text{\cpageref{sec:ggplot2}}$
require(ggplot2)
ggplot(diamonds, aes(depth, fill = cut)) +
  + geom_density(alpha = 0.2) + xlim(55, 70)
# stäng $\text{\TikZ}$-filen
dev.off()
		\end{rcode*}
	\end{minipage}
	\caption{\Rlogo-koden nederst genererar den \PGFTikZ-bild som
	syns överst.}
	\label{ex:tikzdevice}
\end{kod}

\subsubsubsection*{Från \MATLAB med \texttt{matlab2tikz}}
\label{sec:matlab2tikz}
Då \MATLAB, i likhet med \Rlogo, inte har något inbyggt verktyg för att
exportera figurer i \PGFTikZ-format får man även i det fallet förlita sig
på öppen programvara från tredje part. I fallet \MATLAB finns det utmärkta
\texttt{matlab2tikz}\footnote{\url{https://github.com/nschloe/matlab2tikz}},
som implementerats som en enkel funktion som anropas direkt efter det
att man skapat en figur — användningen framgår tydligt av \cref{ex:matlab2tikz}.

Eftersom \texttt{matlab2tikz} är skriven i \MATLAB-kod räcker det i
teorin med att kopiera dess \texttt{.m}-filer till katalogen
\MATLAB-skripten man skriver ligger i, men det kan vara fördelaktigt att
placera filerna i en av de kataloger \MATLAB alltid läser in
\parencite{MATLAB13:path}.

\begin{kod}[tbp]
	\centering
	\begin{minipage}{\textwidth}
		\centering
		\input{\latexbokFiguredir/4/matlab2tikz.tex}
	\end{minipage}
	\\[2ex]
	\begin{minipage}{\textwidth}
		\begin{matlabcode}
% skapa lite exempeldata
[x, y] = meshgrid(-2:0.2:2,-2:0.2:2);
R = (1./((x+y).^2+(y-x).^2+1).^(1/2));
% plotta figuren
ribbon(R);
xlabel('$x$'); ylabel('$y$'); zlabel('$z$');
% spara figuren i tikz-format
matlab2tikz('figur.tex', 'width', '8cm', 'height', '6cm');
		\end{matlabcode}
	\end{minipage}
	\caption{\MATLAB-koden nederst genererar den \PGFTikZ-bild som
	syns överst.}
	\label{ex:matlab2tikz}
\end{kod}

\subsubsection{Exportera i \PDF eller \PNG-fomat}
I de fall då det inte finns något enkelt sätt att exportera figurer som
\PGFTikZ-kod, eller i de fall då det är olämpligt (ger orimligt långa
kompileringstider, till exempel) är \PDF och \PNG relativt bra alternativ.
Båda dessa format är oftast rasterbaserade, men med \PDF finns i alla 
fall möjligheten (och den utnyttjas ibland) att figurerna exporteras
som vektorgrafik.

\subsubsubsection*{Från \Rlogo till \PDF}
\Rlogo har ett antal inbyggda så kallade enheter \eng{device} som kan
användas för att rita diagram, grafer och liknande, bland annat för \EPS,
\PNG och \textsc{SVG} \parencite[\ppno~675–676]{RCoreTeam12}. Enheten vi är mest
intresserade av är den som producerar \PDF-filer. Denna används enkelt
genom att köra kommandot \texttt{pdf}, som tar samma parametrar som
\texttt{tikz}-kommandot som beskrevs tidigare. \PDF-filen som genereras
kan sedan enkelt inkluderas i ett \LaTeX-dokument med \cmd{includegraphics}.
\Cref{ex:pdfdevice} visar hur det kan se ut.

\begin{kod}[tbp]
	\centering
	\begin{minipage}{\textwidth}
		\centering
		\includegraphics{\latexbokFiguredir/4/pdfdevice.pdf}
	\end{minipage}
	\\[1ex]
	\begin{minipage}{\textwidth}
		\begin{rcode*}{mathescape=true}
# width och height specificeras i tum
pdf('figur.tex', width=4, height=2.5)
# exempel med ggplot2, som diskuteras på $\text{\cpageref{sec:ggplot2}}$
require(ggplot2)
ggplot(diamonds, aes(depth, fill = cut)) +
  + geom_density(alpha = 0.2) + xlim(55, 70)
# stäng $\text{\PDF}$-filen
dev.off()
		\end{rcode*}
	\end{minipage}
	\caption{\Rlogo-koden nederst genererar den \PDF-bild som
	syns överst.}
	\label{ex:pdfdevice}
\end{kod}

\subsubsubsection*{Från \MATLAB till \PNG}
I \MATLAB finns ett par olika sätt att exportera figurer, men det
enklaste sättet att spara figurer programmatiskt är med hjälp av 
funktionen \texttt{print} \parencite{MATLAB13:print}. Funktionen kan,
trots sitt namn, spara figurer till ett antal olika format (tyvärr inga
vektorbaserade), där det som oftast fungerar bäst är \PNG.
\Cref{ex:matlab:png} visar hur detta kan se ut i praktiken.

I sitt grundutförande lägger \MATLAB till ganska stora marginaler till
sina figurer. Det är praktiskt när de visas på skärmen, eftersom det ger
lite luft mellan innehållet och fönstrets kant, men det är mest till
besvär när man ska inkludera figuren i \LaTeX. Lyckligtvis är detta
problemet lätt att lösa \parencite{Robertson10} och man kan till exempel
använda funktionen \texttt{figuresize.m}\footnote{\url{https://github.com/wspr/matlabpkg/blob/master/figuresize.m}}, som är fritt tillgänglig under BSD-licensen.

\begin{kod}[tbp]
	\centering
	\begin{minipage}{\textwidth}
		\centering
		\includegraphics{\latexbokFiguredir/4/matlab.png}
	\end{minipage}
	\\[2ex]
	\begin{minipage}{\textwidth}
		\begin{matlabcode}
% skapa lite exempeldata
[x, y] = meshgrid(-2:0.2:2,-2:0.2:2);
R = (1./((x+y).^2+(y-x).^2+1).^(1/2));
% plotta figuren
ribbon(R);
xlabel('x'); ylabel('y'); zlabel('z');
% skala om och spara figuren i PNG-format
figuresize(8, 6, 'centimeters');
print('-dpng', 'figur.png');
		\end{matlabcode}
	\end{minipage}
	\caption{\MATLAB-koden nederst genererar den \PNG-bild som
	syns överst.}
	\label{ex:matlab:png}
\end{kod}
\index{figur!exportera|)}

\label{sec:4:end}
\end{document}
