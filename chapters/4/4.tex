\documentclass[../../latex.tex]{subfiles}
\begin{document}
\section{Grafik med \LaTeX}\label{sec:4}

\subsection{Vad är bra grafik?}
% Se #26 på bitbucket

\subsection{Inkludera grafik}
\subsubsubsection{Varför hamnar figuren fel?}
\subsubsection{Paketet \pack{graphicx}}
\subsubsection{Rita med \PGFTikZ}

\subsection{Exportera grafik}
\subsubsection{Exportera i \PGFTikZ-format}
\subsubsubsection{Från \Rlogo med \texttt{tikzDevice}}
% Att tänka på: ggplot2
\subsubsubsection{Från \MATLAB med \texttt{matlab2tikz}}
%\subsubsubsection{Från \Mathematica med ???}
\subsubsubsection{Från \gnuplot med \TikZ-terminalen}

\subsubsection{Exportera i \PDF eller \PNG-fomat}
% Förklara; PDF>PNG eftersom vektorgrafik
\subsubsubsection{Från \Rlogo till \PDF}
% Att tänka på: ggplot2
\subsubsubsection{Från \MATLAB till \PNG}
% Att tänka på: http://tex.stackexchange.com/questions/5559/how-to-avoid-large-margins-around-matlab-plot-in-pdf
\subsubsubsection{Från \Mathematica till \PDF}
% http://mathematica.stackexchange.com/a/750/1519
\subsubsubsection{Från \gnuplot till \PDF}

\iffalse
Många dokument kräver att man inkluderar figurer för att presentera
resultat eller förtydliga resonemang. Dessa typsätts i \LaTeX{} med hjälp
av flytande objekt, vilket diskuteras i del~\ref{sec:2}, men i dessa
flytande objekt måste man även importera eller generera sin grafik.

\LaTeX{} ger genom diverse paket många möjligheter att göra detta. Med den
klassiska \LaTeX{}-motorn, som genererar \DVI-filer, kunde man importera
\EPS-filer eller rita direkt med \textsc{PostScript}-kommandon, men
eftersom man i dagsläget oftast använder (och bör använda) \pdfLaTeX{}
måste man använda något annorlunda metoder. Denna del förklarar två av
dessa: \pack{graphicx} och \PGFTikZ.

\subsection{Inkludera grafik med \pack{graphicx}}
Det enklaste sättet att inkludera grafik är att skapa sina figurer i ett
externt program (det kan till exempel vara fördelaktigt att exportera sina
figurer direkt från den programvara man använder) och sedan importera
dessa till \LaTeX{}-dokumentet. Detta gör man med paketet \pack{graphicx}.

Proceduren är enkel:
\begin{enumerate}
	\item Skapa figuren i ett externt program och spara i rätt format
	
	\item Inkludera \pack{graphicx}-paketet genom att skriva
	\cmd{usepackage\{graphicx\}} i inledningen till dokumentet.
	
	\item Inkludera grafiken i en \env{figure}-omgivning med hjälp av
	kommandot \cmd{includegraphics}:
	\latex|\includegraphics[parametrar]{filnamn}|
	
	Parametrarna här kan vara många. Oftast vill man ställa in bredden på
	sin bild så att den passar sidan, och detta kan man göra genom att
	specificera parametern \cli{width} och sätta den till
	\cmd{textwidth}:
	\latex|\includegraphics[width=\textwidth]{filnamn}|
	
	Andra parametrar som kan användas är \cli{height}, \cli{angle}
	och \cli{trim}\footnote{Man bör då inte heller glömma att sätta
	\cli{clip=true}.\hfil}. En komplett beskrivning av paketet och de 
	parametrar som finns att tillgå finns i dokumentationen på
	CTAN\footnote{\url{http://www.ctan.org/archive/macros/latex/required/graphics/grfguide.pdf}}.
\end{enumerate}

Säg att man till exempel vill inkludera filen \cli{interpolation.png}
i sitt dokument för att illustrera konceptet interpolation. Man bestämmer
sig dock för att figuren bara ska ta upp 25\,\% av textbredden, eftersom
figuren bli för stor annars. Man använder då parametern \cli{width}
enligt figur~\vref{fig:graphicx}.% för att specificera detta.

\begin{figure}[tbp]
	\centering
	\hspace{-0.025\textwidth}
	\begin{minipage}{0.475\textwidth} % kod
			%\vfil\cprotect[mm]\ifdraft{\inputminted[frame=single,firstline=10,lastline=17,gobble=2]{latex}{chapters/../examples/4/graphicx.tex}}{\inputminted[frame=single,bgcolor=mintedbg,rulecolor=\color{mintedbg},firstline=10,lastline=17,gobble=2]{latex}{chapters/../examples/4/graphicx.tex}}\vfil
	\end{minipage}
	\hfil
	\begin{minipage}{0.475\textwidth} % figur
		\fbox{\includegraphics[width=\textwidth,clip=true,trim=80 0 80 0]{chapters/../examples/4/graphicx.pdf}}
	\end{minipage}
	\caption{Koden till vänster inkluderar filen \cli{interpolation.png} i
	\LaTeX-dokumentet och ger det resultat som ses till höger.}
	\label{fig:graphicx}
\end{figure}

\subsubsection{Rätt format?}
En nackdel med \pdfLaTeX{} när det gäller \pack{graphicx} är att det inte
finns särskilt många bildformat som fungerar (dock fler än för gamla
\LaTeX{}, som bara accepterade \EPS). De format som kan inkluderas
med \pack{graphicx} när man använder \pdfLaTeX{} är \textsc{PNG},
\textsc{JPEG} och \PDF, och oftast vill man använda \PDF{} för figurer
eftersom
det är det enda vektorbaserade formatet som fungerar, medan man för
fotografier och dylikt använder \textsc{JPEG}. \textsc{PNG} kan man
använda då man egentligen bör använda \PDF{} men detta inte är möjligt.

Eftersom många verktyg och programvaror fortfarande sparar filer i
\EPS-format kan det vara praktiskt att kunna konvertera dessa till ett
format som fungerar. Detta kan göras med verktyget \cli{epstopdf}, som
finns tillgängligt på CTAN.

\subsection{Rita med \PGFTikZ}
Ett lite mer komplicerat sätt att inkludera grafik (men ofta 
föredelaktigt, eftersom allt innehåll stannar i \TeX-filen) är att använda
\PGFTikZ, ett paket skrivet för att användas med \pdfLaTeX{} för att rita
figurer i \LaTeX{}. Med det kan man direkt i sitt \LaTeX-dokument skapa
enklare figurer som flödesscheman, träd, grafer eller liknande; se exempel
i figur~\vref{fig:tikz}. Även mer avancerade figurer är möjliga, men att
förklara hela \PGFTikZ{} tar allt för lång tid för denna introduktion. Den
intresserade läsaren hänvisas istället till \citeasnoun{Mertz07} som har
en bra introduktion till ämnet, och \PGFTikZ-manualen \cite{Tantau10} som
utförligt beskriver hur paketet fungerar.

\begin{figure}[p]
	\centering
	\begin{minipage}{0.95\textwidth} % kod
		\vfil
		\begin{latexcode}
\newcounter{d}\setcounter{d}{0}
\def\mcolor{SpringGreen}
\begin{tikzpicture}
\path[coordinate] (0,0)
     coordinate(A) ++( 60:6cm)
     coordinate(B) ++(-60:6cm) coordinate(C);
\draw[fill=Black!\thed!\mcolor]
     (A) -- (B) -- (C) -- cycle;
\foreach \x in {1,...,15}{%
    \pgfmathsetcounter{d}{\thed+10}
    \setcounter{d}{\thed}
    \path[coordinate] coordinate(X) at (A){};
    \path[coordinate] (A)
         -- (B) coordinate[pos=.15](A)
         -- (C) coordinate[pos=.15](B)
         -- (X) coordinate[pos=.15](C);
    \draw[fill=Black!\thed!\mcolor]
         (A)--(B)--(C)--cycle;
}
\end{tikzpicture}
		\end{latexcode}
		\vfil
	\end{minipage}
	\\[1ex]
	\begin{minipage}{0.95\textwidth} % figur
		\centering
		\newcounter{density}\setcounter{density}{0}
		\begin{minipage}{0.475\textwidth}
		\fbox{\begin{tikzpicture}
		    \path[coordinate] (0,0)  coordinate(A)
		                ++( 60:0.95\textwidth) coordinate(B)
		                ++(-60:0.95\textwidth) coordinate(C);
		    \draw[fill=Black!\thedensity!SpringGreen]
						(A) -- (B) -- (C) -- cycle;
		    \foreach \x in {1,...,15}{%
		        \pgfmathsetcounter{density}{\thedensity+10}
		        \setcounter{density}{\thedensity}
		        \path[coordinate] coordinate(X) at (A){};
		        \path[coordinate] (A) -- (B) coordinate[pos=.15](A)
		                            -- (C) coordinate[pos=.15](B)
		                            -- (X) coordinate[pos=.15](C);
		        \draw[fill=Black!\thedensity!SpringGreen]
						(A)--(B)--(C)--cycle;
		    }
		\end{tikzpicture}}
		\end{minipage}\hfil
		\begin{minipage}{0.475\textwidth}
		\caption[Koden ovan tolkas av \PGFTikZ{} och resulterar i den
		itererade triangeln till vänster]{
		Koden ovan tolkas av \PGFTikZ{} och resulterar i den
		itererade triangeln till vänster.
		Fler exempel finns i
		\citeasnoun{Mertz07} och \citeasnoun{Tantau10}. Tack till Alain 
		Matthes för originalkoden\footnote{\url{http://www.texample.net/tikz/examples/rotated-triangle/}}.}
		\label{fig:tikz}
		\end{minipage}
	\end{minipage}
\end{figure}
\fi
\label{sec:4:end}
\end{document}