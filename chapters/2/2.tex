\ifdefined\latexbokFontdir\else\def\latexbokFontdir{../../fonts}\fi
\ifdefined\latexbokFiguredir\else\def\latexbokFiguredir{../../examples}\fi
\documentclass[10pt,../../a4.tex]{subfiles}
\begin{document}
\chapter{Typsättning med \LaTeX}\label{sec:2}
När du nu vet hur den grundläggande strukturen i ett \LaTeX-dokument ser
ut är det dags att börja typsätta riktiga dokument. Att typsätta dokument
är dock något som
tål att tänkas på — den här delen kommer därför inte bara gå igenom
vanliga \LaTeX-konstruktioner och kommandon, utan även att inleda med en
kort diskussion gällande struktur och språk. Efter den här delen bör du
med andra ord inte ha några problem med att producera riktiga dokument,
varken \LaTeX- eller strukturmässigt.

\section{Text- och språkstruktur}
\index{struktur!text-|(}
Struktur är mycket viktigt för att läsaren ska kunna ta del av de idéer
och den information texten förmedlar. Strukturen ska givetvis framgå
direkt ur texten, men med väl genomtänkt typografi förstärks denna
struktur och hjälper läsaren (och författaren) en bit på vägen.

\LaTeX{} skiljer sig från andra typsättningssystem (och ordbehandlare) på
det sättet att du endast behöver berätta för \LaTeX{} vilken logisk och
semantisk struktur din text har. Dessutom så främjar det en god struktur
genom det sätt språket är uppbyggt. När man berättat för \LaTeX{} vilken
struktur texten har tolkar kompilatorn dessa förklaringar efter
förbestämda regler (givna av dokumentklassen och de paket som används) och
skapar den typografiska strukturen utifrån dessa. Du får alltså vara
författaren, och \LaTeX{} tar rollen som typsättare.

\subsection{Textstycket}
\index{styckesindelning}
Den mest grundläggande byggstenen i \LaTeX{} (och både typografi och 
författande) är stycket. Vi kallar det ”byggsten” eftersom ett stycke är
en typografisk enhet som ska förmedla en tanke eller en idé. Om en ny
tanke påbörjas ska alltså även ett nytt stycke påbörjas, och om så inte är
fallet ska endast radbrytningar användas\footnote{Notera dock att det
oftast är helt meningslöst att bry sig om radbrytningar i \LaTeX, eftersom
systemet både avstavar och radbryter bra på egen hand.}. Tvekar du på om
du borde ha ett nytt stycke, tänk på din text som en förmedlare av tankar;
om du har en styckesindelning men din gamla tanke fortsätter, så ska den
troligtvis bort, och om du påbörjar en ny tanke mitt i ett stycke så ska
det förmodligen brytas.

Många underskattar vikten av välgenomtänkta styckesindelningar. Andra vet
inte ens vad syftet med en styckesindelning är, eller (speciellt i 
\LaTeX{}) introducerar nya stycken utan att vara medvetna om det. Detta
misstag är mycket lätt att göra om till exempel ekvationer används i 
texten. Ta
en snabb titt på \cref{ex:stycken} och försök lista ut varför
tomma rader (det vill säga ett nytt stycke) används i vissa tillfällen 
men inte andra — det är en subtil men mycket viktig skillnad.

\begin{kod}[p]
	\begin{latexcode}
% Exempel ett
\ldots{}med hjälp av identiteten
\begin{equation}
\int\limits_{-1}^1\!\sqrt{1-x^2}\;\mathrm{d}x
= \frac{\pi}{2}
\end{equation}
kan man alltså med Monte Carlo-integration ta fram
ett approximativt värde på \(\pi\).

% Exempel två
\ldots{}Weibullfördelningen, döpt efter den svenska
matematikern Waloddi Weibull, har den kumulativa
fördelningsfunktionen
\begin{equation}
F(x) = 1 - e^{-(x/\alpha)^\beta} \;.
\end{equation}

Från denna kan vi med derivering\ldots{}

% Exempel tre
\ldots{}relativt ointressanta observationer.

\begin{equation}
z_{n+1} = z_n^2 + c
\end{equation}
ger å andra sidan upphov till den mycket kända
Mandelbrot-mängden, som\ldots{}

	\end{latexcode}
	\caption{Tre exempel på korrekt styckesindelning i samband med
	ekvationer.}
	\label{ex:stycken}
\end{kod}

Nästa enhet i sammanhanget är en mening. I engelska texter används ett
extra stort mellanrum i slutet av varje mening; detta gör vi inte i
svenskan\footnote{Det finns ganska många saker \LaTeX{} gör som inte bör
göras i svensk typografi — det mesta kommer att tas upp senare i den här 
delen.} och eftersom \LaTeX{} försöker lista ut var den ska lägga in
större mellanrum måste man berätta för \LaTeX{} att den ska låta bli,
vilket görs
med hjälp av kommandot \cmd{frenchspacing}. Skriver man på engelska ska
man alltså utelämna detta kommando, och \LaTeX{} kommer att efter bästa
förmåga gissa var dina meningar tar slut. Gör kompilatorn fel får man 
ersätta det mellanslag den förlänger med ett fast mellaslag (\verb|~|).

\subsection{Rubriker}
\index{rubrik}
Det andra strukturelementet man måste tänka på\footnote{Undantaget
strukturen i meningar, som är den del av svenska språket och som förklaras
bättre av \textcite{LIU98}.} är rubriker. Dessa skapar en logisk
struktur som delar upp innehållet, och den typografiska effekten av att
införa en rubrik är så stark att det nästan är självklart hur man ska
använda dessa för att strukturera sin text.

Standardklasserna i \LaTeX{} har sex rubriknivåer:
\begin{latexcode}
\part{...}          % Ingen nivå
\section{...}       % Nivå 1
\subsection{...}    % Nivå 2
\subsubsection{...} % Nivå 3
\paragraph{...}     % Nivå 4
\subparagraph{...}  % Nivå 5
\end{latexcode}

Dessutom finns det i \pack{report}- och \pack{book}-klasserna en
rubriknivå till, nivå noll, som ges av \cmd{chapter}. 
Eftersom \pack{article}-klassen inte känner igen \cmd{chapter} kan man
enkelt inkludera artiklar som kapitel i till exempel en bok.

Rubriknivån som ges av \cmd{part} påverkar inte de andra numreringarna,
medan alla andra rubrikkommandon återställer numreringen av lägre nivåer.
Detta gör att man kan dela upp sitt dokument i delar, som
\LaTeX{} dessutom per automatik numrerar ner till en viss nivå. Denna
definieras av dokumentklassen men kan ändras genom att ställa in
den inbyggda räknaren \texttt{secnumdepth}:
\latex|\setcounter{secnumdepth}{2} % Numrera ner till subsection|
Dessutom kan man införa onumrerade rubriker genom att lägga till en
stjärna efter rubrikkommandot:
\latex|\section*{Inledning} % Onumrerad rubrik|
Notera även att man inte bör lämna någon tom rad mellan sin rubrik och
det efterföljande textstycket.

Dessa rubriker kan sedan om så önskas listas i en innehållsförteckning med
hjälp av kommandot \cmd{tableofcontents}, där endast de numrerade
rubrikerna inkluderas\footnote{Vill man inkludera även onumrerade rubriker
i sin innehållsförteckning måste man uttryckligen lägga till dessa med
kommandot \cmd{addcontentsline}. Detta läggs lämpligen till på raden efter
rubriken.}.
Djupet på innehållsförteckningen styrs med räknaren \texttt{tocdepth}:
\latex|\setcounter{tocdepth}{2}|

\subsection{Rad- och sidbrytningar}
\index{sidbrytning}
\index{radbrytning!manuell|see{styckesindelning}}
Med \LaTeX{} behöver man i princip aldrig bry sig om att rad- eller
sidbryta eftersom detta görs automatiskt. Vid vissa tillfällen kan det
dock vara önskvärt att göra det manuellt ändå; framför allt kan det vara
intressant att lägga in manuella sidbrytningar för att undvika änkor, 
horungar eller störande avbrott i textflödet. \Cref{tab:newpage}
listar några av de kommandon som kan användas för att åstadkomma
sidbrytningar.

Notera dock att \LaTeX{} inte lyder kommandoföljder à la
”\cmd{newpage}\cmd{newpage}” då sidbrytningskommandon endast inforgar en
sidbrytning \emph{om det behövs}. Vill man ha en tom sida måste man alltså
fylla den med något (osynligt), till exempel en \cmd{mbox} innan man
sidbryter en andra gång.

\begin{table}[tbp]
	\centering 
	\caption{Tre kommandon för att skapa sidbrytningar i \LaTeX.}
	\label{tab:newpage}
	\begin{tabular}{lp{0.65\textwidth}}
		\toprule 
		Kommando & Resultat \\
		\midrule
		\cmd{newpage} & Skapar en ny sida. \\
		\cmd{clearpage} & Skapar en ny sida och tvingar alla \emph{floats}
		(mer om dessa på \cpageref{sec:floats}) att läggas ut innan den nya 
		sidan. 
		Bättre än \cmd{newpage} om syftet är att bryta för ett nytt 
		kapitel eller liknande. \\
		\cmd{cleardoublepage} & Gör samma sak som \cmd{clearpage}, men ser
		till att den nya sidan är en udda sida, förutsatt att dokumentet
		är tvåsidigt. Gör exakt samma sak som \cmd{clearpage} i 
		enkelsidiga dokument. \\
		\bottomrule
	\end{tabular}
\end{table}

Radbrytningar kan enkelt göras med hjälp av \verb|\\| eller \cmd{newline}
i de fall då de behövs; oftast är styckesindelning fullt tillräckligt.

\subsection{Avstavning}
\index{avstavning}
Avstavning sköts automatiskt av \LaTeX{}, förutsatt att det finns
avstavningsdata för det språk man använder. Denna kan man ladda in med
hjälp av paketet \pack{babel}.

Paketet \pack{babel} följer med i princip alla \LaTeX-installationer. 
Det innehåller avstavningsregler
och översättningar till inbyggda kommandon (rubriker till
innehållsförteckning, referenslista och så vidare) för ett stort antal
språk, däribland svenska och engelska. Det är enkelt att använda
\pack{babel} (paketet förklaras närmre i \cref{sec:6}), till och med
för dokument flera olika språk, och oftast räcker det med att helt
enkelt inkludera paketet med rätt inställning:
\latex|\usepackage[swedish]{babel} % Dokumentet är på svenska|

Inga avstavningspaket är dock kompletta, och skriver man tekniska 
rapporter är risken stor att något ord inte finns med eller avstavas på
fel sätt. Man kan då berätta för \LaTeX{} hur man vill att ordet ska
avstavas med hjälp av kommandot \cmd{hyphenate}. Kommandots argument är
en lista av ord (separerade med mellanslag) som har bindestreck
placerade där \LaTeX{} får bryta ordet. Sålunda kan raden
\latex|\hyphenate{FORTRAN span-ku-le-ra}|
användas för att berätta för \LaTeX{} att ordet ”\textsc{FORTRAN}” inte 
får brytas, medan ordet ”spankulera” får brytas på de angivna 
positionerna.

\subsection{Betoning, med mera}
\index{betoning}
\index{textstil!kursiv}
För att betona text i exempelvis böcker använder man \emph{kursiv} text. I
\LaTeX{} kan man betona text med hjälp av kommandot \cmd{emph}, som i de
flesta fall gör just detta; kursiverar texten. Det finns dock andra
tillfällen då man använder kursiv text, och det är givetvis dumt att i
sådana stycken kursivera för att betona text; i sådana fall använder
\LaTeX{} därför helt vanlig, upprätt text (för nästan alla 
dokumentklasser, i alla fall).

\index{textstil!fetstil}
\index{textstil!kapitäler}
\index{typsnitt!serif}
\index{typsnitt!sans-serif}
\index{typsnitt!icke-proportionellt}
\Cref{tab:betoning} visar utöver kommandot \cmd{emph} som används 
för betoning ett antal andra textförändrande kommandon som kan vara bra 
att
känna till. Exempelvis så finns \cmd{textsc}, som typsätter texten med
kapitäler (”små versaler”), och \cmd{textsf} som typsätter med ett
sans-serif-typsnitt istället för det vanliga serif-typsnittet.	
\begin{table}[tbp]\begin{minipage}{\textwidth}
	\centering
	\caption[]{Viktiga \LaTeX-kommandon för att ändra textstil.}
	\label{tab:betoning}
	\begin{tabular}{lp{0.225\textwidth}p{0.375\textwidth}}
		\toprule 
		\LaTeX-kommando & Resultat & Kommentar \\
		\midrule 
		\cmd[textnormal]{textnormal\{\ldots\}}	&
			\textnormal{Dokumentets standardtypsnitt} &
			{Det typsnitt som används för ”vanlig” text
			 i dokumentet.} \\
		\cmd[emph]{emph\{\ldots\}}		&
			\emph{Betonad text} &
			{Ska alltid användas för betoning då kommandot 
			 automatiskt hanterar nästlade betoningar på ett bra sätt.} \\
		\cmd[textrm]{textrm\{\ldots\}}		&
			\textrm{Serif-typsnitt} &
			\\
		\cmd[textsf]{textsf\{\ldots\}}		&
			\textsf{Sans serif-typsnitt} &
			\\
		\cmd[texttt]{texttt\{\ldots\}}		&
			\texttt{Teletype-typsnitt} &
			{Detta typsnitt har fast bredd och kan användas
			 för att typsätta till exempel kodstycken.} \\
		\cmd[textit]{textit\{\ldots\}}		&
			\textit{Kursiv text} &
			{Använd \cmd{emph} istället för detta 
			 kommando för att betona text.} \\
		\cmd[textbf]{textbf\{\ldots\}}		&
			\textbf{Fetstilt text} &
		 	{Använd \cmd{emph} istället för detta 
			 kommando för att betona text.} \\
		\cmd[textsc]{textsc\{\ldots\}}		&
			\textsc{Kapitäler} &
			\\
		\bottomrule 
	\end{tabular}
\end{minipage}\end{table}

Man bör dock inte använda dessa kommandon direkt, utan istället med hjälp
av \cmd{newcommand} definiera kommandon som beskriver \emph{vad} som
typsätts, inte \emph{hur} det ska typsättas. Säg till exempel att du vill
typsätta namn på \LaTeX-paket med sans-serif-typsnitt. Vi kan då definiera
ett kommando \cmd*{package} som i sin tur tillämpar \cmd{textsf}:
\latex|\newcommand\package[1]{\textsf{#1}}|
När detta sedan används i dokumentet ser man direkt att det är ett paket
som beskrivs; att det sedan typsätts med sans-serif-typsnitt är inte så
intressant just när man skriver dokumentet.
\index{struktur!text-|)}

\section{Figurer, tabeller, listor}
Figurer, tabeller och listor\footnote{Men även ekvationer, programkod,
vissa kemiska formler, citat och algoritmer, för att nämna några
ytterligare exempel.} har i sin \LaTeX-representation en sak gemensamt;
de kräver mer än enkla kommandon för att kunna uttryckas på ett enkelt
sätt. Därför har man i \LaTeX{} valt att införa en konstruktion som
grupperar sådana företeelser och gör det tydligt hur textens struktur
verkligen ser ut — denna konstruktion kan på svenska kallas för en 
omgivning.

\subsection{Omgivningar \eng{environments}}\label{sec:omg}
\index{miljö}
Omgivningar i \LaTeX{} definieras utifrån två kommandon; \cmd{begin} och
\cmd{end}. De kan nästlas (på det sättet att en omgivning kan innehålla
en annan) men kan inte överlappa — varje omgivning måste avslutas innan
dess ”förälder” avslutas, vilket gör att de ger en tydlig trädliknande
struktur. En omgivning (här \texttt{verbatim}) kan alltså skrivas
\begin{latexcode}
\begin{verbatim}
Det här är omgivningens innehåll
\end{verbatim}
\end{latexcode}
där omgivningens innehåll (oftast) med fördel kan indenteras för att göra
strukturen tydlig.

Det finns många omgivningar, och de fyller alla olika syften. De
viktigaste bildar listor, figurer, tabeller och ekvationer, men en
omgivning kan i princip göra vad som helst.

\subsection{Tre sorters listor}
\index{lista}
De första omgivningarna vi ska bekanta oss med är de som skapar listor.
I \LaTeX{} finns det tre användbara sorters listor, som alla går att
nästla i varandra utan problem och som löser tre distinkta problem:
numrerade listor (\texttt{enumerate}), onumrerade listor 
(\texttt{itemize}) och beskrivningslistor (\texttt{description}). Dessa
används tillsammans med kommandot \cmd{item} för att bygga enkla listor
(se \cref{ex:listor}) och är relativt självförklarande.

\begin{kod}[tp]
	\centering
	\cprotect{\subfloat[En numrerad lista]}{
		\begin{minipage}{0.475\textwidth} % kod
			\begin{latexcode}
\begin{enumerate}
\item Skapa en lista
\item Lägg till ett element
\item Upprepa steg 2 om så
    önskas
\end{enumerate}
			\end{latexcode}
		\end{minipage}
		\hfil
		\begin{minipage}{0.475\textwidth} % figur
			\fbox{\parbox{\textwidth}{
				\begin{enumerate}
					\item Skapa en lista
					\item Lägg till ett element
					\item Upprepa steg 2 om så önskas
				\end{enumerate}
			}}
		\end{minipage}
	}\\
	\cprotect{\subfloat[En onumrerad lista]}{
		\begin{minipage}{0.475\textwidth} % kod
			\begin{latexcode}
\begin{itemize}
\item Tyskland
\item Topologi
\item Teknolog
\end{itemize}
			\end{latexcode}
		\end{minipage}
		\hfil
		\begin{minipage}{0.475\textwidth} % figur
			\fbox{\parbox{\textwidth}{
				\begin{itemize}
				  \item Tyskland
				  \item Topologi
				  \item Teknolog
				\end{itemize}
			}}
		\end{minipage}
	}\\
	\cprotect{\subfloat[En beskrivningslista]}{
		\begin{minipage}{0.475\textwidth} % kod
			\begin{latexcode}
\begin{description}
\item[Häst] ett stort,
    fyrbent däggdjur.
\item[Öl] en mycket god
    dryck.
\end{description}
			\end{latexcode}
		\end{minipage}
		\hfil
		\begin{minipage}{0.475\textwidth} % figur
			\fbox{\parbox{\textwidth}{
				\begin{description}
			  \item[Häst] ett stort,
			        fyrbent däggdjur.
			  \item[Öl] en mycket god
			        dryck.
				\end{description}
			}}
		\end{minipage}
	}
	\caption{De tre enkla listomgivningar \LaTeX{} tillhandahåller.}
	\label{ex:listor}
\end{kod}

\subsection{Flytande objekt \eng{floats}}\label{sec:floats}
\index{flytande objekt}
Tabeller och figurer är viktiga verktyg för att förmedla resultat på
ett effektiv sätt. De är också mycket speciella objekt rent typografiskt,
eftersom de till skillnad från vanliga textstycken inte kan brytas över
sidor hur som helst. För att placera figurer och tabeller på ett
tillfredsställande sätt använder \LaTeX{} sig av så kallade \emph{floats},
vilket innebär att objektet får ”flyta iväg” en bit ifrån det ställe där
det från början definieras. Detta undviker fula halvtomma sidor och andra
obehagliga resultat.

I många böcker typsätts figurer och tabeller så att de antingen flyter
till toppen av en sida, botten av en sida, eller så att de tar upp en hel
sida. Man typsätter i princip aldrig figurer och tabeller så att de bryter
texten. Man kan även göra samma sak med till exempel kod, men detta är
inte lika vanligt. Även om det finns sätt att tvinga \LaTeX{} att lägga en
figur där den definieras så rekommenderas detta alltså inte.

Ett flytande objekt definieras av en omgivning, och har ett valfritt
argument som berättar för \LaTeX{} hur objektet får placeras. Detta kan
innehålla bland annat \texttt{t} (\emph{top}), \texttt{p} (\emph{page})
och \texttt{b} (\emph{bottom}) och är en lista över alla de sätt \LaTeX{}
tillåts placera objektet. Om inget av dessa fungerar kommer \LaTeX{} att
placera objektet på en egen sida. Sålunda kan man skapa en \emph{float}
med följande kod:
\begin{latexcode}
\begin{<floattyp>}[tbp] % Får placeras som top eller bottom
% Innehåll i det flytande objektet
\end{<floattyp>}
\end{latexcode}

Flytande objekt kan även ha figurtexter, som definieras med kommandot
\cmd{caption}. En figurtext ska komplettera figuren och förklara vad som
visas, inte varför det visas. Slutsatser och liknande dras istället i
brödtexten som refererar till figuren. Figurer har sin figurtext undertill
medan tabeller har figurtexten ovanför tabellen.

\subsubsection*{Figurer}
\index{figur}
Figurer är den kanske vanligaste typen av flytande objekt, och dessa
(grafer, illustrationer, diagram och så vidare) infogas i omgivningen
\texttt{figure}, vilken visas i \cref{ex:figure}.

\begin{kod}
	\centering\hspace{0.0075\textwidth}
	\begin{minipage}{0.75\textwidth} % kod
			\vfil\ifdraft{\inputminted[frame=single,firstline=10,lastline=16]{latex}{\latexbokFiguredir/2/figure.tex}}{\inputminted[frame=single,firstline=10,lastline=16,bgcolor=mintedbg,rulecolor=\color{mintedbg}]{latex}{\latexbokFiguredir/2/figure.tex}}\vfil
	\end{minipage}
	\\ \medskip
	\begin{minipage}{0.725\textwidth} % figur
		\fbox{\includegraphics[width=\textwidth,clip=true,trim=25 -2 25 -2]{\latexbokFiguredir/2/figure.pdf}}
	\end{minipage}
	\caption{Ett exempel på hur man skapar ett flytande objekt med
	\texttt{figure}.}
	\label{ex:figure}
\end{kod}

Att infoga själva figuren kan göras på några olika sätt; man kan skapa den
med ett externt program (till exempel \cli{gnuplot} eller Inkscape), 
exportera den direkt från MATLAB eller Mathematica, eller till och med
rita den direkt i \LaTeX. Hur man gör detta diskuteras närmre i
\cref{sec:4}.

För att centrera innehållet i figuren använder man kommandot 
\cmd{centering}, som centrerar allt innehåll i resten av omgivningen. En
del får för sig att istället använda omgivningen \texttt{center}, men
eftersom den omgivningen lägger till extra tomrum både före och efter är
det lämpligare att bara använda \cmd{centering}.

\subsubsection*{Tabeller}\label{pack:booktabs}
\index{tabell}
Tabeller är ett annat vanligt flytande objekt, och är ett alldeles
förträffligt verktyg när det gäller presentation av data som till exempel
mätserier eller liknande. \LaTeX{} definierar två tabellrelaterade
omgivningar; \texttt{table} och \texttt{tabular}. Det förstnämnda är ett
flytande objekt och används precis som \texttt{figure}:
\begin{latexcode}
\begin{table}[tpb]
\centering 
\caption{En beskrivning av tabellen}
% Här ska själva tabellen in
\end{table}
\end{latexcode}

Den andra omgivningen, \texttt{tabular}, är den som faktiskt används för
att definiera tabellen. \LaTeX{} kan i denna omgivning hjälpa till med att
rita linjer både kors och tvärs, och definiera kolumner på ett antal olika
sätt, men för att hålla sig till en enkel (om än bokinspirerad) devis så
bör man ta paketet \pack{booktabs} till hjälp när man skapar tabeller i
\LaTeX\footnote{Exakt varför detta är en bra idé förklaras mycket bra av
\pack{booktabs}-manualen \parencite{Fear05}.}.
En enkel tabell skapad med hjälp av \pack{booktabs} kan se ut som
tabellen i \cref{ex:tabular}.

\begin{kod}[tbp]
	\centering 
	\begin{minipage}{0.9\textwidth} % kod
		\begin{latexcode}
\begin{tabular}{l r p{4cm}}
\toprule 
Konstant & Värde & Kort beskrivning \\
\midrule 
\(\gamma\) & \(0,577\) & Skillnaden mellan den
 harmoniska summan från \(1\) till \(n\) och den
 naturliga logaritmen av \(n\) då \(n\to\infty\).\\
\(e\) & \(2,718\) & Den konstant som har
egenskapen att \((e^x)' = e^x\). \\
\(\pi\) & \(3,1415\) & Kvoten mellan en cirkels
 omkrets och dess diameter. Mycket viktig
 konstant i många sammanhang; dyker upp
 lite varstans. \\
\bottomrule
\end{tabular}
		\end{latexcode}
	\end{minipage}
	\\ \medskip
	\fbox{\begin{minipage}{0.9\textwidth} % figur
		\centering \medskip
		\begin{tabular}{l r p{6.5cm}}
			\toprule 
			Konstant & Värde & Kort beskrivning \\
			\midrule 
			\(\gamma\) & \(0,577\) & Skillnaden mellan den
			 harmoniska summan från \(1\) till \(n\) och den
			 naturliga logaritmen av \(n\) då \(n\to\infty\).\\
			\(e\) & \(2,718\) & Den konstant som har
			egenskapen att \((e^x)' = e^x\). \\
			\(\pi\) & \(3,1415\) & Kvoten mellan en cirkels
			 omkrets och dess diameter. Mycket viktig
			 konstant i många sammanhang; dyker upp
			 lite varstans. \\
			\bottomrule
		\end{tabular}
		\medskip
	\end{minipage}}
	\caption{En tabell skapad med hjälp av \pack{booktabs}.}
	\label{ex:tabular}
\end{kod}

Som vi ser tar \texttt{tabular}-omgivningen ett obligatoriskt argument,
som beskriver de kolumner tabellen innehåller. \Cref{tab:kolumner}
listar de vanligaste kolumntyperna och deras användningsområden.
Kolumntyperna anges direkt efter varandra, men kan om man vill separeras
med exempelvis mellanslag. Den speciella separatorn \texttt{|} skapar en
vertikal linje mellan två kolumner, men detta är inte att rekommendera
\parencite{Fear05}.

Efter att omgivningen inletts med \cmd{begin} följer tabellens rader. Här
måste varje rad markeras med explicita nyrader (\verb|\\|), och kolumnerna
separeras med och-tecken (\verb|&|). Notera även de kommandon som skapar
horisontella linjer; dessa är från \pack{booktabs} och används enligt
\cref{ex:tabular}.

Det är även viktigt att komma ihåg några typografiska regler när det
gäller tabeller. Alla dessa ges av \textcite[3]{Fear05}. Fritt
översatt:

\begin{quotation}
	\begin{enumerate}
		\item Använd aldrig någonsin lodräta linjer.
		\item Använd aldrig dubbla linjer.
		\item Lägg enheter i tabellhuvudet.
		\item Decimalkomma ska alltid föregås av en siffra; alltså \(0,1\) 
		\emph{inte} bara \(,1\).
		\item Använd inte upprepningstecken eller liknande konventioner 
		för att repetera föregående värde. I många fall räcker det med en 
		tom cell. Gör det inte det, repetera värdet.
	\end{enumerate}
\end{quotation}

Punkt ett och två är enkla att följa om man använder \pack{booktabs} som
manualen föreskriver, och punkt tre och fem är snarare arbetsbesparande än
ansträngande att hålla sig till. Även punkt fyra är enkel att följa. 
Följer man
dessa fem punkter får man snygga tabeller varje gång\footnote{Läs gärna
igenom \textcite{Fear05} ordentligt de tre-fyra första gångerna du
sitter med tabeller i \LaTeX, så går användandet in i ryggmärgen.}.

\begin{table}[tbp]
	\centering 
	\caption{De fyra viktigaste kolumntyperna i \texttt{tabular}.}
	\label{tab:kolumner}
	\begin{tabular}{l p{0.65\textwidth}}
		\toprule 
		Typ & Kommentar \\
		\midrule 
		\texttt{l} & Vänsterjusterar innehållet. Använd till text, datum
					 och så vidare. \\
		\texttt{c} & Centrerar innehållet. Använd inte. \\
		\texttt{r} & Högerjusterar innehållet. Använd till tal, 
				     mätresultat och så vidare. \\
		\texttt{p\{...\}} & Avstavat textstycke med angiven bredd. Använd
							när texten är så lång att tabellen bli för
							bred för sidan, eller när du skriver långa
							beskrivningar och/eller kommentarer. \\
		\bottomrule 
	\end{tabular}
\end{table}

\subsection{Etiketter och korsreferenser}\label{sec:labels}
\index{etikett}
\index{korsreferens}
Eftersom figurer får (och bör få) flyta en bit bort, i vissa fall till 
andra sidor, måste man kunna referera till dem i texten. Detta görs med så
kallade korsreferenser till etiketter. En etikett kan man definiera med
kommandot \cmd{label}, inte bara för figurer och tabeller utan även för
rubriker, ekvationer och annat. För flytande objekt och andra omgivningar
placerar man \cmd{label}-kommandot inuti omgivningen (men efter 
\cmd{caption} om denna används), medan man för rubriker placerar etiketten
direkt efter \cmd{section} (eller motsvarande).

En etikett kan innehålla många olika tecken, men det kan vara bra att
hålla sig till (amerikanska) bokstäver, siffror samt kolon (\texttt{:}) 
som separator.
För att kunna hålla koll på sina etiketter bör man även namnge dem på ett
logiskt sätt (varje etikett bör beskriva det den refererar till) och gärna
inleda varje etikett med en liten beskrivande förkortning. Man kan till
exempel märka en figur som visar \(\beta\)-sönderfall med följande etikett:
\latex|\label{fig:betasonderfall}|

Etiketterna kan man sedan referera till med \cmd{ref}, som skriver ut det
nummer figuren (eller motsvarande) har. Dessutom finns \cmd{pageref}, som
skriver ut vilken sida figuren ligger på. Dessa kommandon är dock inte
medvetna om vilken typ av objekt man refererar till, så detta måste man
berätta:
\latex|...som man ser i figur~\ref{fig:betasonderfall}...|

Paketet \pack{varioref} inför det något mer intelligenta kommandot
\cmd{vref} (och även \cmd{vpageref}) som formulerar referensen på ett
något bättre sätt, beroende på var figuren ligger. Ligger figuren på samma
sida är \cmd{vref} i princip ekvivalent med \cmd{ref}, men om figuren är
på en annan sida skriver \cmd{vref} även ut vilken sida detta är.

Ett annat paket, som kan ses som en förbättring av \pack{varioref}, är
\pack{cleveref}. Detta paket räknar automatisk ut vilken typ av referens
det är du refererar till, och skriver ut detta. Exemplet ovan kan då
reduceras till detta:
\latex|...som man ser i \cref{fig:betasonderfall}...|

\index{hyperlänk}
Paketet \pack{hyperref} förbättrar alla dessa kommandon genom att
göra referenserna till länkar i den slutgiltiga \PDF-filen. Med dessa kan
man alltså lättare navigera i dokumentet. Förutom detta så lägger
\pack{hyperref} dessutom till en innehållsförteckning som \PDF-läsaren
kan visa i programmet, och annan \PDF-specifik funktionalitet. Paketet
rekommenderas starkt om du använder \pdfLaTeX.

\subsection{Ordlistor}\label{sec:2:glossaries}
\index{ordlista}
När man skriver tekniska eller vetenskapliga texter finns det ofta
anledning att inkludera en ordlista. Det väldigt omfattande paketet
\pack{glossaries} löser precis denna uppgift.

Paketet, som dokumenteras väldigt utförligt av \textcite{Talbot13a},
är mycket omfattande och kan användas för att skapa många sorters
ordlistor och förteckningar. Den kortare introduktionen för nybörjare
\parencite{Talbot13b} går igenom hur man använder paketet för att skapa
en enkel ordlista eller lista över förkortningar.

Den som är intresserad av att använda \pack{glossaries} bör läsa 
\textcite{Talbot13b}, men i princip går det ut på att man deklarerar en
ordlista med kommandot \cmd{makeglossaries}, definierar termer till
ordlistan med \cmd{newglossaryentry} och refererar till dessa med
\cmd{gls}, \cmd{glspl}, \cmd{Gls} och \cmd{Glspl}. Därefter skriver man
ut själva ordlistan med kommandot \cmd{printglossaries}.

Paketet använder även programmet \cli{xindy} för att sortera ordlistan,
och därför måste man (i likhet med \pack{biblatex} som förklaras i
\cref{sec:5}) köra \cli{pdflatex}, \cli{xindy} och sedan \cli{pdflatex}
igen (eller \cli{makeglossaries} istället för \cli{xindy}, om du har
Perl installerat). Detta förklaras närmre av
\textcite{Talbot13a,Talbot13b}.

\subsection{Förkortningar}\label{sec:2:acro}
\index{förkortning}
I viss teknisk litteratur (avhandlingar, artiklar och liknande) kan det 
även vara bra att inkludera en lista över förkortningar. Det problemet
går att lösa med \pack{glossaries}, men det är ofta bättre att använda
ett specialiserat paket, till exempel \pack{acro}
\parencite{Niederberger13}. Till skillnad
från \pack{glossaries} använder \pack{acro} inte några externa program,
och är därför något enklare och mindre tidskrävande att använda.

\subsection{Sakregister}\label{sec:2:imakeidx}
\index{register}
En sista komponent i många större projekt, speciellt böcker eller
liknande litteratur, är ett sakregister. Även dessa kan skapas på ett
relativt automatisk sätt av \LaTeX, med hjälp av paketet \pack{imakeidx}.
Paketet förklaras av \textcite{Beccari12}, men användningen liknar
på många sätt \pack{glossaries}. I likhet med \pack{glossaries} måste man
köra \cli{xindy} eller \cli{makeindex} mellan \LaTeX-körningarna, men 
under vissa förhållanden\footnote{Om man kör \cli{pdflatex} med flaggan
\cli{--shell-escape}, eller om \TeX-distributionen är konfigurerad med
restriktiv \cmd{write18}-åtkomst som tillåter dessa program.} kan \LaTeX\ 
göra detta automatiskt. Givetvis görs detta även av \cli{latexmk} om man
använder det verktyget för att kompilera sina dokument.

\section{Typografiska betänkligheter}
När man skriver rapporter är det även viktigt att hålla sig till de
typografiska regler \LaTeX{} inte håller reda på; sådana som har med
citattecken, tankstreck och annat att göra. Även datum, decimalavskiljare
och liknande är viktigt att tänka på. Dessa typografiska regler kommer i
korthet att förklaras på de närmsta sidorna, till vilka viss inspiration 
tagits från \textcite{Schultz05}.

\subsection{Citattecken}
\index{citattecken}
Det mest grundläggande man bör uppmärksamma i \LaTeX{}, eftersom det är en
grop man lätt faller i, är citattecken. Man kan inte i \LaTeX{} använda
det vanliga citattecknet (\verb|"|) för att generera ett citattecken, utan
man måste istället använda kombinationer av \emph{backticks} (\verb|`|)
och apostrofer (\verb|'|). \Cref{tab:citat} visar hur dessa används
för att citera text.

\begin{table}[tbp]
	\centering 
	\caption{Citattecken i \LaTeX.}
	\label{tab:citat}
	\begin{tabular}{l l p{0.5\textwidth}}
		\toprule 
		\LaTeX-kod & Resultat & Kommentar \\ 
		\midrule 
		\verb|``Text''| & “Text” & Engelska citattecken. Använd inte i
		svensk text! \\
		\verb|`Text'| & ‘Text’ & Nästlade engelska citattecken (för citat
		inuti citat). Använd inte i svensk text! \\ 
		\verb|''Text''| & ”Text” & Svenska citattecken. \\
		\verb|'Text'| & ’Text’ & Nästlade svenska citattecken (för citat
		inuti citat). \\
		\bottomrule 
	\end{tabular}
\end{table}

\subsection{Streck av olika längd}
\index{bindestreck}
\index{tankstreck}
\index{intervallstreck}
Det kan tyckas vara ett rent typografinörderi man inte behöver bry sig om,
men faktum är att det finns olika sorters streck (lite på samma sätt som
att det finns olika skiljetecken). Det är relativt viktigt att skilja på
dessa och se till att använda rätt streck vid rätt tillfälle.

Det första strecket, minustecknet, är relativt långt (\(-\)) och går i
princip inte att missa eller missbruka. Det infogas av \LaTeX då man
använder ett vanligt bindestreck (\verb|-|) i matematikläge\footnote{
Mer om detta i \cref{sec:3}.}:
\latex|...och ett flyttal kan därför ha värdet \(-0\).|

Nästa tecken, det vanliga bindestrecket (-), behöver man i princip aldrig
använda själv. Det infogas av \LaTeX{} då ord avstavas, men kan även
infogas manuellt genom att helt enkelt skriva ett vanligt bindestreck
i \LaTeX{}-koden.

Det tredje strecket kallas intervallstreck (–) och är något längre än
bindestrecket. Detta används som namnet antyder för att typsätta
intervall, och skrivs med hjälp av två bindestreck (\verb|--|):
\latex|...det kan ta upp emot 3--4 timmar att...|

Det fjärde och sista strecket kallas långt tankstreck (—) och är det längsta av
strecken. Det långa tankstrecket är relativt ovanligt inom svensk typografi, men det
finns ingen anledning att undvika det\footnote{För tankar (\emph{no pun
intended}) angående användningen av tankstreck, se 
\textcite[\ppno~46–47]{LIU98} — observera dock att man \emph{inte} som 
\citeauthor*{LIU98} föreslår ska använda tankstrecket i intervall eller
punktlistor. Det \textcite{Sprakradet08} och \citeauthor*{LIU98} betecknar som
”tankstreck” är i själva verket mycket närmre det som i \LaTeX{} anses vara ett
intervallstreck, det vill säga ett lite kortare tecken. Personligen föredrar jag 
att göra skillnad på de två, och gör man det bör intervallstrecket användas till
intervall, och det långa tankstrecket som alternativ till kommatecknet.}.
Ett långt tankstreck skriver man med tre bindestreck i följd
(\verb|---|), och man bör omge det med mellanslag:
\latex|...det är --- för tillfället åtminstone --- omöjligt att...|

\subsection{Avstånd mellan stycken}
\index{styckesindelning}
Det finns två vedertagna sätt att separera stycken; indrag och
mellanrum. Man tillämpar endast \emph{ett} av dessa åt gången, och
traditionellt sett så brukar man i svensk typografi använda mellanrum,
medan man i amerikansk typografi använder indrag (varför detta är vad
\LaTeX{} gör \emph{out-of-the-box}).

För att få \LaTeX{} att använda mellanrum istället för indrag bör man
egentligen skriva helt nya dokumentklasser, men eftersom detta är ett
stort jobb kan man istället ”\eng{patcha}” \LaTeX{}. Detta görs av
paketet \pack{parskip}:
\latex|\usepackage{parskip}|

\subsection{Andra saker att hålla reda på}
\begin{itemize}
\item Datum skrivs på bästa sätt i ISO-format (YYYY–MM–DD), där man
	använder intervallstreck för att skilja delarna åt.
	
\item Decimalavskiljare i svensk skrift är kommatecknet. I matteläge
	tolkat \LaTeX{} ett komma som en koordinatavskiljare, och lägger
	till ett litet mellanrum efter. För att skriva ut numeriska värden
	med kommatecken som avskiljare kan man använda kommandot \cmd{num}
	från \pack{siunitx}, som förklaras närmre på \cpageref{sec:3:siunitx}.
	
\item Enheter ska typsättas med ett (litet) mellanrum mellan tal och
	enhet. Detta görs enklast med paketet \pack{siunitx}.
\end{itemize}

\subsection{Skarpare typsnitt: \pack{fontenc} och \pack{lmodern}}
\index{typsnitt}
\index{typsnitt!skärpa}
\label{sec:2:lmodern}
Eftersom \LaTeX{} och dess tillhörande typsnitt (Computer Modern) är
relativt gamla, och eftersom de designats för att fungera väl med det
mycket gamla \textsc{PostScript}-formatet, så uppstår det ibland några
smärre problem. Det första problemet som brukar dyka upp i samband med
typsnitt i \LaTeX{} är att texten blir otydlig och suddig i det typsatta
\PDF-dokumentet. Detta beror på att de typsnitt som används är baserade
på bitmappar, som inte skalar särskilt väl. Därför måste man använda
paketet \pack{fontenc} för att säga åt \LaTeX{} att använda vektorbaserade
typsnitt istället:
\latex|\usepackage[T1]{fontenc}|

Det är då man brukar stöta på nästa problem, som yttrar sig i att text
(främst skandinaviska tecken och andra tecken som inte finns 
representerade i \textsc{ANSI}) inte kopieras på ett korrekt sätt. Detta
är en konsekvens av att de mycket gamla Computer Modern-typsnitten inte
innehåller de glyfer som motsvarar sådana tecken, och istället används då
(något fulhackade) komposita tecken istället. För att lösa detta måste man
byta typsnitt.

Det typsnitt som brukar användas istället för Computer Modern är dess
mer moderna klon \emph{Latin Modern}. Detta typsnitt ser
i princip ut som Computer Modern, men innehåller alla de glyfer som behövs
för att typsätta svensk (och annan icke-anglikansk) text. För att använda
Latin Modern inkluderar man paketet \pack{lmodern}:
\latex|\usepackage{lmodern}|

\section{Andra viktiga delar}
Även om innehållet är det viktigaste så finns det andra delar av en
rapport eller artikel som också måste ges lite uppmärksamhet.
Titelsidor bör man till exempel alltid ha, och det är även viktigt att ha 
koll på
hur man citerar och vilken teckenkodning man använder. I vissa fall
kan man även vilja referera till vissa längder, eller använda kommandon
som \LaTeX{} anser vara ”ömtåliga” i situationer de inte är tänkta att
användas. Avslutningen på detta kapitel kommer därför att gå igenom dessa
småsaker.

\subsection{Titelsidan}% +abstracts, sammanfattning och båda samtidigt
\index{titelsida}
Det kan vara trevligt att i sitt arbete ha en fin titelsida. Även denna
kan \LaTeX{} (åtminstone standardklasserna) typsätta åt dig. Detta är
enkelt att göra, och fungerar i princip så att man definierar sin
metadata (titel, datum, författare) med hjälp av ett par kommandon, varpå
man kör kommandot \cmd{maketitle} för att skriva ut en titelsida:
\begin{latexcode}
\title{En fantastisk rapport\thanks{Tack till DD-gudarna
   för visad barmhärtighet och stabila filservrar.}}
\date{2011--12--13}
\author{E.~Johansson\thanks{erik.johansson@example.org}}
\maketitle
\end{latexcode}

Kommandot \cmd{thanks} kan användas för att infoga korta tack eller extra
information om titel eller författare, så som mailadresser eller
liknande. Den typografiska effekten av \cmd{thanks} (i standardklasserna)
än en fotnot.

Man kan dessutom infoga en sammanfattning \eng{abstract} med hjälp av
omgivningen \texttt{abstract}. Denna omgivning inkluderas lämpligen direkt
efter \cmd{maketitle}, och skapar automatisk ett litet textstycke med
rubriken ”Sammanfattning” (eller, om engelska valts som språk,
”Abstract”).

Vill man (säg, i ett kandidatarbete) ha både \emph{abstract} och
sammanfattning måste man använda \pack{babel}s funktioner för att byta
språk mitt i dokumentet. Man måste då först ladda in \pack{babel} med stöd 
för båda språken (notera att det sista språket i listan är aktivt från
dokumentets start):
\latex|\usepackage[english,swedish]{babel}|

	Därefter byter man helt enkelt språk när man vill ha sitt \emph{abstract}:
\begin{latexcode}
\begin{otherlanguage}{english}
\begin{abstract}
    % Abstract
\end{abstract}
\end{otherlanguage}
\begin{abstract}
% Sammanfattning
\end{abstract}
\end{latexcode}

\subsection{Citat}
\index{citattecken}
% FIXME: Korrekturläs och förbättra
I \LaTeX{} används ofta \verb|``| och \verb|''| för att typsätta citattecken,
och dessa teckenkombinationer resulterar då i det öppnande och stängande
citattecknet så som det används i engelsk typografi. I svensk typografi
används endast det senare, så skriver du på svenska ska du bara använda
\verb|''|. Använder man UTF-8 i sina dokument (vilket man bör) kan man
även använda citattecknet \verb|”| självt (notera att detta inte är det
tecken som normalt produceras med Shift+2 utan ett korrekt citattecken).

Det finns även två omgivningar i \LaTeX{} som kan användas för att märka upp
så kallade blockcitat \eng{blockquotes}, alltså lite längre citat på ett
eller flera stycken; \texttt{quote} och \texttt{quotation}. Det första,
\texttt{quote}, är tänkt att användas till citat på högst ett stycke,
medan \texttt{quotation} ska användas för flera stycken.

För att det här med citattecken, längre citat och även källhänvisning ska
kunna hanteras på ett enklare och mer konsekvent sätt skapades paketet
\pack{csquotes}. Det är ett paket som varmt rekommenderas och som inför
ett antal användbara kommandon. Bland annat definieras kommandot
\cmd{enquote}, som automatiskt omsluter texten med rätt citattecken även
för citat inuti citat, och
\cmd{textquote}, som är en variant som även lägger till en källhänvisning.
Dessutom definieras omgivningen \env{displayquote}, som ersätter både
\env{quote} och \env{quotation}. Paketets dokumentation, som finns på CTAN,
är värd en genomläsning då paketet har många användbara kommandon.

\subsection{Fotnoter}\label{sec:2:footnote}
\index{fotnot}
Fotnoter är små noteringar eller parantetiska inlägg%
\footnote{Ibland används även fotnoter för källhänvisning — mer om detta
i \cref{sec:5}.} som inte är väsentliga men som författaren ändå
vill inkludera utan att störa textrytmen. I \LaTeX{} kan man skapa 
fotnoter med kommandot \cmd{footnote}:
\latex|\ldots{}ibland vill\foonote{Eller måste, beroende på situation} man\ldots|

Om man använder fotnoter bör man även använda paketet \pack{fnpct}, som
förutom att fixa fotnoter nära skiljetecken även definierar ett kommando
\cmd{multfootnote}, som gör det möjligt att typsätta flera fotnoter brevid
varandra.

\subsection{Teckenkodning (\UTF)}
\index{teckenkodning}
Teckenkodning, som kort nämndes på \cpageref{sec:1:inputenx}, är
viktigt att hålla koll på. Det finns två större teckenkodningar på Linux-%
plattformar, \UTF{} och \textsc{iso-8859-1}, och en till på Windows-%
system, \textsc{cp-1252}. Det är viktigt att man vet vilken teckenkodning
man skriver sina dokument i, då man måste berätta detta för \LaTeX. Gör
man inte det förutsätter \LaTeX{} att dokumentet är skrivet i
\textsc{iso-8859-1}, och är det då inte det så kommer hemska saker hända.
Teckenkodningen specificeras med paketet \pack{inputenx}%
\footnote{Tidigare användes \pack{inputenc}, men det har ersatts av det
mer uppdaterade \pack{inputenx}.}:
\latex|\usepackage[<teckenkodning>]{inputenx}|
Här refererar man till \UTF{} som \texttt{utf8}, \textsc{iso-8859-1} som
\texttt{latin1} och \textsc{CP-1252} som \texttt{cp1252}. Det finns även
en del mer obskyra teckenkodningar\footnote{De som stöds av 
\pack{inputenx} listas i paketets manual.}, men det är troligtvis de tre
nämnda du kommer att stöta på.

\subsection{Längder}
\index{längd}
Titt som tätt behöver man i \LaTeX{} använda längder för att till exempel
kontrollera bredden på figurer och liknande. Den viktigaste fördefinierade
längden är \cmd{textwidth}, som representerar bredden av det område på
sidan som får fyllas med text. Genom att lägga till ett flyttal innan
längden kan man dessutom berätta för \LaTeX{} att man till exempel vill
ha en figur som endast täcker \num{3/4} av bredden:
\latex|0.75\textwidth|

Det finns en stor mängd längder i \LaTeX, och många av dem relaterar till
olika mått på sidan \parencite[132]{Oetiker11}, medan andra (till exempel
\cmd{smallskipamount}) är mer obskyra och godtyckliga.

\subsection{Ömtåliga kommandon}
En del text, så som rubriker, figurtexter med mera, kan dyka upp mer än en
gång i ett dokument (till exempel i innehållsförteckningen). Argument till
sådana kommandon kallas för rörliga \eng{moving arguments} och kan ställa
till med en del problem. Det finns nämligen kommandon (\cmd{footnote},
till exempel) som är ”ömtåliga” \eng{fragile}, och som därför inte kan
användas i rörliga argument utan vidare. Vill man göra det måste man
skydda dem, något som görs med kommandot \cmd{protect}:
\latex|\section{En rubrik\protect\footnote{Med en fotnot.}}|
\end{document}
