\RequirePackage[l2tabu]{nag}
\documentclass[a4paper,11pt]{scrartcl}
% Allmängiltiga paket
\usepackage[utf8]{inputenx}
\usepackage[T1]{fontenc}
\usepackage{lmodern}
\usepackage[swedish]{babel}
\usepackage{microtype}
\usepackage{icomma}
\usepackage{graphicx}
\usepackage{fixltx2e}
% Matematikpaket
\usepackage[intlimits]{amsmath}
\usepackage{amssymb}
\usepackage{mathtools}
%\usepackage{siunitx} % finns ej på Chalmers-datorer
\usepackage{textcomp}
% Snyggare tabeller
\usepackage{booktabs}
% Snyggare figurtexter
\usepackage[font=small,%
            format=plain,%
            labelfont=bf,%
            textfont=it]{caption}
% Länkade och intelligenta referenser
\usepackage{hyperref}
\usepackage{cleveref}
% Källförteckning
\usepackage[style=authoryear]{biblatex}

% Inställningar för siunitx
%\sisetup{locale=DE}

% Kommandodefinitioner (finns även i paketet skmath)
% \abs{}, \norm{} — absolutbelopp och norm
% \frac omdefinierad för att se bättre ut
% \d{} — integraldifferentialen
\makeatletter
\DeclarePairedDelimiter\abs{\lvert}{\rvert} 
\DeclarePairedDelimiter\norm{\lVert}{\rVert}
\renewcommand{\frac}[2]{\genfrac{}{}{}{}%
             {\displaystyle #1}{\displaystyle #2}}     
\renewcommand\d[1]{\ensuremath{\;\mathrm{d}#1%
                   \@ifnextchar\d{\!}{}}}
\makeatother

% Dokumentets metadata
\title{Ett exempeldokument}
\author{Simon Sigurdhsson}
\date{6 augusti 2013}

% Här börjar dokumentet
\begin{document}
    \maketitle 
    \begin{abstract}
        Man bör alltid ha en kort sammanfattning (eng.
        \emph{abstract}) till sina rapporter och
        artiklar. En sådan skapas med omgivningen
        \texttt{abstract}.
    \end{abstract}
    
    % En innehållsförteckning vill man ha ibland:
    \tableofcontents
    \newpage
    
    \section*{Introduktion}
    Det här avsnittet är tänkt att innehålla en kort
    introduktion till rapporten som förklarar dess syfte
    och mål. Man behöver så klart inte alltid en
    introduktion, men om man har en så bör den vara
    onumrerad, och ska således skapas med
    \texttt{\textbackslash section*}.
    
    \section{Första avsnittet}
    Som du ser kan man radbryta sin text i \LaTeX{} utan
    att det påverkar resultatet. Gör det --- det blir mer
    lättläst. Vissa textredigerare kan göra detta
    automatiskt.
    
    Det fungerar i ekvationer också:
    \begin{equation}
        f(x) = \sin x
        \implies 
        F(x) = \int\! f(x)\d{x} = \cos x + c.
    \end{equation}
    
    \subsection{Ett underavsnitt}
    Man kan även infoga tabeller i \LaTeX{} (se tabell
    \ref{tab:priser}, till exempel) och figurer (figur
    \ref{fig:rulltarta}). Dessa kan man referera till på olika
    sätt, till exempel med kommandot
    \texttt{\textbackslash cref}: ``\cref{fig:rulltarta}''.
    
    \begin{table}[tp]
        \centering 
        \caption{En tabell över fiktiva priser}
        \label{tab:priser}
        \begin{tabular}{lp{0.4\textwidth}r}
            \toprule 
            Namn & Beskrivning & Pris (\$) \\
            \midrule 
            Cykel & Enkelt tvåhjuligt fordon utan motor
            för transport av enstaka personer över korta
            sträckor & 200 \\
            Polkagris & Röd-vit-randig sötsak med mycket
            hård konsistens, tillverkad i Gränna & 2,50\\
            \bottomrule 
        \end{tabular}
    \end{table}
            
    \begin{figure}[bp]
        \centering 
%        \includegraphics[width=0.8\textwidth%
%                         ]{bilder/rulltarta.png}
        \caption{En mycket god rulltårta}
        \label{fig:rulltarta}
    \end{figure}
\end{document}
